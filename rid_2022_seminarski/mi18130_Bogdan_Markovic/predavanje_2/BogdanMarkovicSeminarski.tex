% !TEX encoding = UTF-8 Unicode

\documentclass[a4paper]{article}

\usepackage{color}
\usepackage{url}
\usepackage[T2A]{fontenc} % enable Cyrillic fonts
\usepackage[utf8]{inputenc} % make weird characters work
\usepackage{graphicx}
\usepackage{placeins}
%\graphicspath{ {./slike/} }
%\documentclass{article}
\usepackage[utf8]{inputenc}
\usepackage[serbian]{babel}
\usepackage{graphicx}
\usepackage{amsmath}
\newtheorem{exercise}{\Large Zadatak}
\usepackage[export]{adjustbox}
\usepackage{caption}
\usepackage{subcaption}
\usepackage{fancyhdr}
\usepackage[bottom]{footmisc}
\pagestyle{fancy}
\usepackage[figurename=Слика]{caption}
\usepackage[serbian]{babel}
\fancyhf{}
\usepackage{gensymb} 
\author{Богдан Марковић}
\usepackage{fancyhdr}
\pagestyle{fancy}
\makeatletter
\let\runauthor\@author
\let\runtitle\@title
\makeatother
\lhead{\runauthor}
\rhead{Филтерски мехур}
%\usepackage[english,serbianc]{babel} %ukljuciti babel sa ovim opcijama, umesto gornjim, ukoliko se koristi cirilica

\usepackage[unicode]{hyperref}
\hypersetup{colorlinks,citecolor=green,filecolor=green,linkcolor=blue,urlcolor=blue}

%\newtheorem{primer}{Пример}[section] %ćirilični primer
\newtheorem{primer}{Primer}[section]

\begin{document}

\title{Филтерски мехур\\ 

\bigskip
\small{Семинарски рад
\\Рачунарство и друштво
\\ Математички факултет}}

\author{Богдан Марковић}
\date{септембар, 2022.}
\maketitle



\newpage

\tableofcontents

\newpage

\section{Увод}
\label{sec:uvod}
Енглески професор, филозоф и теоретичар медија Херберт Маршал Маклуан (1911–1980), још је давне 1962. године предвидео да ћемо имати интернет. Тада је у својој књизи \emph{Гутенбергова галаксија} приметио да се људска историја може поделити на четири поглавља: акустичко доба, књи-
жевно доба, доба штампања и, тада у зачетку, електронско доба\cite{macluhan}. Електрон-
ско доба поништава однос центра и периферије. Све постаје могуће, на било којем месту на планети, а види се и чује кад год се на планети догоди неки значајан чин. Глобално село је за Маклуана нов ниво поновног спајања у органску целину „механизованих делића фрагментиране цивилизације“. Маклуан је веровао да ће ново доба бити дом за оно што је назвао „глобално село“ – простор у којем технологија шири информације свуда и са свима. Ниједна Маклуанова фраза није имала толико успеха као „глобално село”. Први пут је уведена у рукопису \emph{Извештај о пројекту и разумевању медија} из 1960, а онда је представљена свету у књизи \emph{Гутенбергова галаксија} две године касније. Фраза је била толико популарна да је пронашла место и у наслову две Маклуанове књиге – \emph{Рат и мир у глобалном селу} из 1968. године, те у посмртно објављеној књизи \emph{Глобално село} (1989).

„Од појаве телеграфа и радија планета се просторно ограничила у једно велико село. Трибализам\footnote{Трибализам (енгл. tribalism од лат. tribus – племе) соц. 1. у примитивним друштвима, организација друштва на нивоу племена. 2. у савременом политичком систему, организација заснована на солидарности људи који потичу из исте етничке групе, краја и
сл.} је наш једини извор од електромагнетског открића. Прелазак са принта у електронске медије значи да смо дали око за уво“, запажа аутор у својој књизи \emph{Разумевање медија}. Међутим, Маклуан се није бавио само предностима мреже. Он је
видовито упозорио да би предаја „приватној манипулацији“ ограничила обим наших информација на основу садржаја који оглашивачи бирају за кориснике. Нажалост, умро је неколико година пре него што је могао да види како се његова предвиђања остварују, али доласком интернета Маклуан је изронио из заборава и поново се јавило интересовање за његов рад, поготово за идеју да масовни медији играју важну улогу у обликовању свести човека.

Живети у ери дигиталне револуције је привилегија савременог човека. Интернет и глобална повезаност, лак приступ информацијама и људима већ одавно је свет свео на глобално село. Умрежавање читаве планете био је пројекат подржан од многих држава, великих корпорација и већине станов-
ништва, јер је потенцијал вишеструк, а корист и имплементација очигледни. Посао, забава, саобраћај, криминал, безбедност - готово свака пора живота проналази свој дигитални пандан на интернету. Стога многи аутори истичу да је интернет несумњиво један од најпровокативнијих социолошких и пси-
холошких феномена. Поједини проучаваоци овог феномена сматрају да интернет представља слободну мрежу на којој се повезују „посрнули комуни-
катори“ који се не сналазе у реалном свету. Међутим, слободно дружење и сурфовање на онлајн простору не пролази без остављања трагова. Сваки корисник даје податке о себи, некад као услов да би могао да приступи траженим апликацијама, када свесно на то пристаје, некад размењујући податке са другим учесницима мреже, несвестан да мрежа „све памти”, да подаци остају депоновани чак и у случају када уклоне свој профил.



\newpage
\section{Лични подаци на интернету}
\label{sec:naslov1} 

Право на заштиту личних података и приватности јесте једно од основних људских права. Са наглим развојем дигиталне технологије и интернета, ово право озбиљно је доведено у питање. У времену које карактерише обрада огромне количине најразличитијих података, тзв. „ери великих података”, лични подаци третирају се као „нова нафта”. Своје „бесплатне” услуге компа-
није наплаћују корисницима тако што им заузврат траже све више личних података, јер, како каже стара изрека, „бесплатан ручак не постоји”. Подаци о корисницима, њиховим активностима и понашању на интернету користе се за анализу и креирање социјално-психолошких профила, циљано пласирање комерцијалних производа прилагођених индивидуалним карактеристикама и потребама корисника („један-на-један маркетинг”), за продају компанијама или сервисима („трећим лицима”), итд. Према речима Хелен Нисенбаум (Helen Nissenbaum), професорке информатике на Универзитету Корнел, приватност није право на тајност, нити на контролу, већ право на одговара-
јући протокол личних података. То значи да особа има могућност да, у зависности од ситуације и контекста, лично процени шта ће и са ким делити у дигиталном окружењу. Или, другачије речено, има право да зна како и у које сврхе се користе њени подаци, ко их чува и колико дуго, ко све њима располаже, као и да затражи брисање личних података или исправку нетачних података.

\begin{figure}[hbt!]
\centering
\includegraphics[width=8cm]{data.jpg}
\caption{Различити извори података }
\end{figure}

\subsection{Подела личних података на интернету}
\label{subsec:podnaslov1}

Лични подаци на интернету могу се груписати у три категорије:

\begin{itemize}
    \item Активни дигитални трагови – подаци (о себи или другима) које сами корисници остављају приликом коришћења интернета, обично свесно, мада не нужно и намерно (нпр. приликом куповине неких производа, преузимања нечега са интернета, постављања фотографија, отварања профила на некој друштвеној мрежи)
    \item Пасивни дигитални трагови – подаци које корисници остављају на интернету приликом његовог коришћења, углавном несвесно (нпр. путем колачића, отиска прстију, података о локацији, коришћења паметних ствари...)
    \item Подаци добијени анализом првих двеју категорија података, помоћу алгоритама (кроз процес профилисања), евентуално у комбинацији са другим изворима података.
\end{itemize}

Бројне личне информације могу бити скривене у фотографијама или видео-садржајима које постављамо на интернет, као и у разговору путем чета или имејла.



\subsection{Проблеми}
\label{subsec:podnaslov2}


Све чешће смо суочени са крађама података и идентитета од стране криминалних хакерских група. Ова дела углавном пролазе некажњено јер је интернет, као „слободна” мрежа, често изван законске контроле. Сајбер безбедност је изазов који ће морати регулаторно да се реши у времену које долази. То доводи до дуализма у решавању ове проблематике. Проблем настаје због могућности да се решавањем питања безбедности с једне стране, наруши заштита приватности на интернету с друге стране. Непромишљено остављање личних података на различитим сајтовима представља огромни маневарски простор за манипулацију људима, финансијске злоупотребе и повреде приватности. Надзирући интересовања кроз персонализоване пре-
траге, велике интернет компаније имају могућност да корисницима, сходно њиховим интересовањима, сервирају сличан, али често непожељан и непотре-
бан садржај. Таква неселективност у одабирима, тај филтерски мехур којим бивамо обавијени, неретко доводи до озбиљних поремећаја личности: памћења, концентрације, емоционалне и опште отупљености, такозване дигиталне де-
менције.

Ширење нових средстава комуникације и нових технологија које се користе да би се обезбедила њихова функционалност уводи у раније непознате појаве и дефиниције, чија се суштина не открива одмах. Као последица јавља се осећај збуњености, а да би се схватило о чему је реч, урања се у дубине интернета. Међутим, што се дубље иде, виртуелна стварност постаје све компликованија.


\newpage
\section{Шта је филтерски мехур?}
\label{sec:naslov2}

\subsection{Појава и узроци}
\label{subsec:podnaslov3}

Појава филтерског мехура, који сужава опсег корисничког информа-
ционог окружења, једна је од непредвиђених последица Интернета које највише забрињава. Питање је, међутим, како настају ови мехури полариза-
ције. Да ли су они својствени алгоритамским механизмима филтрирања, или настају и из других узрока? Сви алгоритамски системи се могу разумети у смислу три елемента: података, алгоритамске логике и људске интеракције. Апстраховањем на овај начин избегавате да будете ухваћени у сложеност и разноврсност техника науке о подацима. Такође је противтежа природној тенденцији да се фокусира искључиво на саме алгоритме. Док услуге као што је Фејсбук доприносе филтерском мехуру алгоритамском препоруком садржаја који јача постојеће ставове, шта корисници деле за почетак и на шта кликну када алгоритам изађе на површину такође је веома важно. Симулациони експеримент показује како се филтерски мехури појављују или колабирају из интеракције сва три фактора.

\begin{figure}[hbt!]
\centering
\includegraphics[width=8cm]{filter-bubbles.jpg}
\caption{Илустрација филтерског мехура}
\end{figure}

Ако корисник примети како сајтови са вестима брзо нуде корисне инфор-
мације, онда је, а да то није знао, упао у филтерски или информациони мехур. Може се рећи да се сваки корисник налази унутар „информационог балона“. Ели Парисер, активиста и аутор књиге \emph{The Filter Bubble: What the Internet is Hiding From You} (2011), истиче како се веб компаније боре да скроје своје услуге (укључујући вести и резултате претраге) у складу са нашим личним укусима, у чему постоји опасна намерна последица: бивамо заробљени унутар мехура и остајемо ускраћени за информације које би прошириле наш поглед на свет. Ели Парисер упозорава да је ова тенденција усмерена против човека и демократије јер интернет контролише и, пружа-
јући информације, истовремено ограничава у информацијама.\cite{pariser}

\subsection{Различите дефиниције филтерског мехура}
\label{subsec:podnaslov4}

Кембриџ речник нуди ову дефиницију филтерског мехура: „Ситуација у којој неко само чује или види вести и информације које подржавају оно у шта већ верује и оно што воли”\cite{dictionary}. Слободна енциклопедија Википедија овај термин приписује активисти Парисеру који је истакао да је „филтерски мехур или идеолошки оквир стање интелектуалне изолације које може бити резултат персонализованих претрага када алгоритам веб-сајта селективно погоди које информације би корисник желео да види на основу самих информација о кориснику, као што су локација, прошлост кликтања и историја претраге. Као резултат, корисници постају одвојени од информација које се не слажу са њиховим гледиштима, ефективно их изолујући у њиховим сопственим културним или идеолошким мехурима. Избори које доносе ови алгоритми нису увек транспарентни. Најбољи примери укључују Гугл персонализоване резултате претраге и Фејсбукову персонализовану насловну страну“.\cite{wiki}

Технологија као што су друштвени медији „омогућава вам да се дружите са људима истог мишљења, тако да не мешате, делите и разумете друге тачке гледишта... То је веома важно. Испоставило се, међутим, да је то већи проблем него што сам ја, или многи други, очекивали.” (Бил Гејтс, 2017)



\subsection{Опасности од филтерског мехура}
\label{subsec:podnaslov5}

Алгоритми и филтери које користе највеће интернет компаније показују шта је најрелевантније за сваког појединог посетиоца, стварајући својеврсну зону комфора. Било да се ради о прегледавању веба, гледању вести или наручивању робе, компаније као што су Гугл, Фејсбук (Мета) и друге, нуде корисницима информације које се чине најприкладнијим. Парисер даје врло једноставан пример који доказује валидност његове теорије: када двоје људи у Гугл претраживач укуца само једну реч – назив државе, први добија информације о политичкој структури те земље, док други – о туристичким рутама. Изоловани корисник у ограниченом информационом простору, често не слути да постоје друге могућности информација, избора и гледишта. „Чим се повежете на интернет, алгоритми платформи које користите прате шта вам привлачи пажњу“, наглашава Парисер. Они нуде кориснику садржај пошто закључе шта му се допада и шта ће, вероватно, конзумирати. Овај процес доводи до стварања филтерског мехура. Многи су стручњаци ову појаву означили као „онлајн манипулацију”, односно покушај утицаја на доношење одлука и понашања појединца, контролу и управљање људима помоћу информационих технологија\cite{susser}. Ова манипулација подразумева на-
мерни и прикривени утицај на доношење одлука људи, а да они тога нису свесни. У својој књизи Парисер открива како савремени интернет функцио-
нише и упозорава на опасности изолације у мехуру филтера. Алгоритми за претрагу користе мноштво корисничких података да би пронашли и представили релевантне информације сваком појединачном посетиоцу, увек када приступи интернету. Историја претраге је кључна за прилагођавање резултата које добијате када претражујете веб. С једне стране, ови алгоритми играју улогу навигатора у океану информација и помажу да се брзо пронађе област и подаци за које би иначе требало много времена, а које савременом човеку недостаје. Међутим, наглашавањем само појединих чињеница и пода-
така алгоритамски системи одсецају друге области, лишавајући корисника не само битних информација, већ и могућег добијања објективне слике са разликама у мишљењима, алтернативним изборима и важним закључцима.

\begin{figure}[hbt!]
\centering
\includegraphics[width=8cm]{diagram.png}
\caption{Резултати алгоритамских система могу се приписати њиховим основним подацима, математичкој логици алгоритама и начину на који људи комуницирају са овим факторима. (извор: \url{https://www.cambridge.org/core/books})}
\end{figure}

Бити унутар информационог балона може имати различите последице за различите групе људи. На пример, понављајуће, наметљиво оглашавање производа и услуга може или подстаћи куповину или изазвати фрустрацију и узнемиравање, што ће утицати на лично расположење или буџет појединца или породице. Сајтови за е-трговину већ дуго користе системе препорука за филтрирање, сортирање и предлагање својих производа, као и медија које користимо на интернету. Ствари се компликују када се утицај алгоритама прошири на јавне сфере, посебно када је реч, на пример, о политичким дискусијама и феномену политичке поларизације, расним или верским разли-
кама у процени друштвеног статуса, родним преференцијама у објављивању огласа за посао, итд. Пружање „искривљених“ информација, као што је, на пример, понуда искључиво конзервативних вести конзервативним корисн-
ицима и либералних вести либералним корисницима, могу довести до гре-
шака и сукоба на нивоу друштва, до конфронтације друштвених група и неуспешних одлука у управљању од стране власти., Забринутост због овог проблема навела је групу америчких сенатора да априла 2019. године предложи Закон о алгоритамској одговорности у којем су истакли да би „аутоматски системи доношења одлука“ могли да испоље пристрасност и дискриминацију, посебно када реч је о питањима као што су приватност и безбедност (Booker, 2019). Међутим, стручњаци сматрају да пребацивање проблема на алгоритме и механизме формирања алгоритама значи игнорисање друге важне компоненте система – самих људи, који дају знатну количину података читавом ланцу деловања унутар система, производећи његову дина-
мику и смер\cite{miller}. Штавише, емпиријске студије које је спровела група стручњака са Оксфорда и Харварда показују да живимо у ехо коморама које стварамо, а алгоритми само незнатно побољшавају ове мехуриће. Притиском на дугмиће „подели“, „свиђа ми се“, „претплати се“, људи са истим мишљењем окупљају се у хомогене групе које деле исте тачке гледишта и прикупљају информације из истих извора. У зависности од скупа идеја и информационих података, такве ехо коморе могу произвести и значајан позитиван набој, али и негативне последице, о чему сведоче различите студије које су спровели стручњаци о утицају информационих технологија на друштво.



\subsection{Примери манипулације. Алгоритми и друштвене мреже}
\label{subsec:podnaslov6}

Када корисник посети одређену веб страницу, мисли да добија исти садржај као и сви остали. Али, скоро свуда на мрежи алгоритми прате шта корисник жели да кликне како би пружали садржај на основу онога што мисле да се њему допада, и наставиће то да раде све док не покажу садржај који ће он вероватно конзумирати. Овај процес може довести до стварања филтерског мехура. 

Јутјуб вам предлаже који следећи видео-клип да погледате, Дизер коју песму да слушате, ИМДБ у ком бисте филму могли да уживате, а Фејсбук приказује нове постове ваших пријатеља и, у највећем броју случајева, они погоде шта ће вас заиста занимати. Осим тога, у данашње време при рекламирању производа важно је приказати их на друштвеним мрежама и то управо оним људима који би могли бити заинтересовани за њих. Иза свега овога стоје алгоритми који се машинским учењем адаптирају на нове ситуације и доносе све прецизније и примамљивије понуде. За лаике је јако тешко да дођу до података како заиста изгледају огољени овакви алгоритми, али и стручњацима је тешко да их прате јер се непрестано усавршавају и развијају нови, а притом се за највећи број операција користи комбинација више алгоритама.

\begin{figure}[hbt!]
\centering
\includegraphics[width=8cm]{urn_cambridge.org_id_binary_20200814151449286-0105_9781108610018_42663fig4_5.png}
\caption{Када два корисника користе исту ствар, дода се веза између њих. Мери се број и пропорција грана између корисника различитих типова. (извор: \url{https://www.cambridge.org/core/books})}
\end{figure}

Циљ сваке друштвене мреже је да пронађе садржај који занима одређеног корисника којим ће га привући да што чешће користи ту друштвену мрежу, а већи број активних корисника донеће и веће могућности за плаћено рекла-
мирање. Тако, на пример, на Фејсбуку имате опцију да означите у ком граду живите, и тиме не само да ћете добити предлоге да за пријатеље додате људе из истог града, већ ћете добијати и рекламе за које су се највише заинтересовали људи из истог града. Уколико сваког од корисника видимо као чвор у мрежи, везе међу њима су јаче ако имају заједничке познанике, па је тако већа вероватноћа да се познају двоје пријатеља једне особе него двоје насумично изабраних људи. У оваквим групама постоји и низ заједничких интересовања, па је унутар групе знатно једноставније циљано рекламирати одређени производ. При рекламирању преко друштве-
них мрежа треба, пре свега, имати у виду како, рецимо, функционише излиставање објава на Инстаграму. Свакога дана особе или странице које пратите на Инстаграму објаве бројне различите фотографије, али ће објаве бити приказане у неком, нимало случајном, редоследу. Оне се рангирају на основу вашег претходног понашања, као и на основу ваше повезаности са другим корисницима, па ће вам се готово сигурно појавити фотографија којој сте "лајковали" последњих десет фотографија, али је мала вероватноћа да ћете видети фотографију особе са којом никада не комуницирате. Видљи-
вост сваког поста, уопштено говорећи, зависи од неколико фактора: колико вам је занимљива особа која га поставља, како је пост прошао код других корисника, коју врсту постова и колико је пост нов. Осим тога, на Инстаграму вам се могу појавити и плаћени огласи, а с обзиром на то да је број огласа које ћете у току дана видети ограничен, програмери се труде да вам кори-
шћењем великих података и алгоритама прикажу управо производ који бисте могли да купите или догађај који бисте волели да посетите.

Алгоритми који персонализују и креирају искуство корисника на мрежи, звучи као добра ствар. Обиље информација на мрежи човек 21. века нема времена да истражује. Осим тога, свако има специфична интересовања, па се фокусира на садржај који ће му дати одговоре на нека питања. На пример, сајт друштвених медија може сакрити постове пријатеља са разли-
читим гледиштима, или веб локација са вестима може приказати чланке са којима корисник мисли да се слаже. Можда чак и не схвата да је у мехуру филтера јер ови алгоритми не траже дозволу, не наговештавају када су активни или шта крију. У ствари, они су постали део интернета у целини, те их је скоро немогуће избећи.

Када корисник изврши претрагу, а затим кликне на линк, термини за претрагу се шаљу на локацију на коју је кликнуо (http протокол). Ово дељење личних података назива се „цурењем претрагe“. Поред тога, када посети било коју локацију, рачунар аутоматски шаље информације на ту локацију (укључујући и IP адресу). Ове информације се често могу користити за директну идентификацију.


\newpage
\section{Покушаји решавања проблема}
\label{sec:naslov3}

Једном када се корисник заглави у свом мехуру, проблем се само погор-
шава. На пример, ако су сви уверени да добијају целовиту причу о актуелном догађају, а заправо добијају само део тога, нико не може донети објективан суд и постаје тешко водити смислену дискусију о чињеницама. Тако мехур доприноси неразумевању и неспремности да се размотре супротстављена гледишта и неповољне информације. Због тога корисник не зна како да се постави према филтерским мехурима и алгоритмима. Неке компаније предузеле су извесне кораке ка решавању овог проблема.

\subsection{Duckduckgo}
\label{podnaslov7}

\emph{Duckduckgo}, америчка компанија за заштиту приватности на интернету фокусирана на безбедност корисника, истиче да омогућава кориснику непри-
метно преузимање контроле над личним подацима на мрежи, без икаквих компро-
миса, чиме спречава цурење претраге. Приватност интернет грађана је свакога дана све више угрожена, па потреба за безбедним начинима претраге константно расте. Претраживач \emph{Duckduckgo} ради од 2008. године, а последњих година је привукао већи број корисника који су на време схватили да им је потребан претраживач који не чува податке. Будући да не злоупотребљава информације о корисницима у маркетиншке сврхе, у оквиру овог претраживача нема реклама ни колачића, а комбинација са ТОР софтвером га чини потпуно анонимним. Алтернатива је свим другим претраживачима, а процењује се да ће постајати све популарнији, будући да спречава злоупотребу података. Овај претраживач не користи податке са Гугла, а корисници сваки пут „крећу од нуле“, будући да се не чува ни историја претраживања, нити било који други подаци уз помоћ којих се на другим претраживачима креирају кориснички профили. Уместо тога, када се кликне на линк на сајту, овај претраживач усмерава (преусме-
рава) тај захтев на такав начин да не шаље термине за претрагу корисника на друге сајтове. Други сајтови ће и даље знати да их је корисник посетио, али не и коју претрагу је претходно унео. 

\subsection{Гугл}
\label{podnaslov8}

Гугл је увео онлајн образац за пријаву за Европљане који желе да се лични подаци уклоне из резултата онлајн претраге. Компанија је преплав-
љена захтевима за уклањање након што је највиши европски суд подржао ово право. Случај је покренуо
Шпанац Марио Гонзалес, који се пожалио да је обавештење о аукцији његовог одузетог дома у резултатима Гугл претраге нарушило његову приватност. Захтеви за уклањање података наводно су поднети након што је Европски суд правде донео одлуку да појединац може захтевати да се неважне или застареле информације избришу из резултата, јер је приватност приоритет у дигиталном добу. Компанија Гугл образац за пријаву сматра „почетним напором“ у решавању захтева за заштиту података, али за сада нема дефинитивног решења овог проблема. До тада сваки корисник треба имати на уму филтерски мехур док претражује интер-
нет, јер ће на тај начин моћи делимично да контролише своје податке на мрежи.


\newpage
\section{Закључак}
\label{sec:zakljucak}

Живот савременог човека је тешко замислив без медија. Масовне комуни-
кације су постале доминантан начин размене информација, али и сервис који пружа неопходна сазнања за свакодневни живот. Масовни медији омогу-
ћили су да се простор „премости“ и да информација о догађају буде доступна у моменту његовог одвијања, без обзира на удаљеност од реципијента. Си-
стеми за масовну дистрибуцију информација све се више глобализују - све је мање простора до кога не допире емитована информација. Тиме су се медији исказали као веома моћно средство комуникације, чији потенцијал може бити усмерен на добробит свих, али и искоришћен за разне облике манипулације. Велики број истраживача овог феномена сматра да су данашњи медији основна средства психоманипулације, јер је њихов развој довео до динамичне манипулације. Многи теоретичари комуникације су мишљења да је утицај медија на људе (на свесни и несвесни део личности) већи него што бисмо хтели да признамо (Аћимовић, 2009). Они подстичу разнолике активности и усмеравају реакције (како прогресивне и хумане, тако и деструк-
тивне и нехумане) појединаца и група. Тај утицај може бити више или мање директан или индиректан, имплицитан или експлицитан, очигледан или суптилан. Медији су оруђе које они који њиме располажу користе у различитом степену и зарад манипулације.

Очигледно је да феномен филтерског мехура и ехо комора захтева емпи-
ријска
истраживања, с обзиром на то колико су интернет простор и комуни-
кационе технологије
продрле у наш лични и друштвени живот. У овој ситу-
ацији само објективно знање може
да нас спасе и од претераних страхова и од неозбиљног приступа проблему. Остаје питање
ко ће коме да господари: човек технологији или напредна технологија човеком?




\addcontentsline{toc}{section}{Литература}
\appendix

\iffalse
\bibliography{seminarski} 
\bibliographystyle{plain}
\fi

\bigskip
\newpage

\begin{thebibliography}{7}

\bibitem{macluhan} MacLuhan M. \emph{Understanding Media}, McGraw-Hill, 1964.

\bibitem{pariser} Pariser E. \emph{The Filter Bubble: What The Internet Is Hiding From You}, New York Times, 2011.

\bibitem{susser} Susser D., Russler B., Nissenbaum H. \emph{Technology, autonomy and manipulation}, Internet Policy Overview, 2019.

\bibitem{miller} Hosangar K., Miller A.P. \emph{Who Do We Blame for the Filter Bubble? On the Roles of Math, Data, and People in Algorithmic Social Systems}, Cambridge University Press, 2020.

\bibitem{dictionary} \emph{The Cambridge Advanced Learner's Dictionary}, Cambridge University Press, 1995.

\bibitem{echo} \emph{Filter bubble and echo chambers}, Foundation Descartes, 2020.

\bibitem{wiki} Wikipedia \emph{Filter bubble}


\end{thebibliography}


\end{document}
