% !TEX encoding = UTF-8 Unicode

\documentclass[a4paper]{article}

\usepackage{color}
\usepackage{url}
\usepackage[T2A]{fontenc} % enable Cyrillic fonts
\usepackage[utf8]{inputenc} % make weird characters work
\usepackage{graphicx}
\usepackage{amsmath, derivative}

\usepackage[english,serbian]{babel}
%\usepackage[english,serbianc]{babel} %ukljuciti babel sa ovim opcijama, umesto gornjim, ukoliko se koristi cirilica

\usepackage[unicode]{hyperref}
\hypersetup{colorlinks,citecolor=green,filecolor=green,linkcolor=blue,urlcolor=blue}

%\newtheorem{primer}{Пример}[section] %ćirilični primer
\newtheorem{primer}{Primer}[section]

\begin{document}
	
	\title{Društvene mreže i mentalno zdravlje\\ \small{Seminarski rad u okviru kursa\\Računarstvo i društvo,\\ Matematički fakultet}}
	
	\author{Svetlana Bićanin}
	\date{maj 2022.}
	\maketitle
	
	{
		 		
		\tableofcontents
		
		\newpage
		
		\section{Uvod}
		\label{sec:uvod}
		Još od početka ljudske istorije bilo je nemoguće zamisliti društvo koje ne koristi različita sredstava ljudske komunikacije. Ovo se posebno odnosi na savremeno društvo. Različiti tipovi medija oblikuju individualni razvoj, razmišljanje, stavove i ponašanja, kako na pozitivan tako i negativan način.
		Internet, sa ulaskom hiljada novih korisnika iz celog sveta, nesumnjivo dovodi do nove socijalne revolucije. To predstavlja vrhunac digitalne industrijske revolucije, a svaka revolucija posledično dovodi do potencijalnih problema.
		Jedinstvene karakteristike interneta, kao što su 24-časovna dostupnost, jednostavan rad, pristupačnost, potencijalna anonimnost korisnika i sl, pozdravlju korisnici širom sveta, ali postavlja se pitanje da li su to pozitivne ili pak negativne karakteristike. Ostavlja se prostor za diskusiju kakav je uticaj ostvarilo širenje društvenih mreža na čoveka kao socijalno i umno biće. Svi smo svesni benefita koje društvene mreže nose sa sobom, ali smo i sve više svesni posledica poput ugrožavanja svog fizičkog i mentalnog zdravlja, srljajući sve više u očigledan dizbalans. U zadnjih par godina smo svedoci sve većeg negativnog uticaja. Treba istaći i to da je prekomerna upotreba društvenih mreža velika zabrinutost, prvenstveno za roditelje.
		
		\section{Pregled kroz istoriju}
		Istorijski posmatrano, godine  \textbf{1994.} kreiran je sajt \textbf{Geocities}, preteča današnjih društvenih mreža. Sajt je dozvoljavao korisnicima da kreiraju svoje veb-stranice, grupišući ih u skladu sa sadržajem. Nakon njega, nastali su novi sajtovi – SixDegrees, Classmates, Hi5, Friendzy. Popularnost društvenih mreža je rasla sve više i više, a to je dovelo i do sledećeg velikog stupnja u razvoju ovog internet fenomena. Prva moderna društvena mreža je bio Friendster. Veoma sličan današnjem konceptu društvenih mreža Friendster je koristio koncept „krugova“ prijatelja i u roku
		od tri meseca od svog nastanka imao je 3.000.000 korisnika. Nakon toga usledio je MySpace, društvena mreža koja je dominirala internetom od 2005. do 2008. godine. Juna 2006. godine ta mreža je u Sjedinjenim Američkim Državama imala više poseta nego Google.
		\subsection{Facebook}
		\label{subsec:podnaslov1}
		Fejsbuk(Facebook) je na internet istupio 2004. godine i za samo četiri godine uspeo je da osvoji svet i postane najveća društvena mreža. Uspešnim akvizicijama(WhatsApp, Instagram, Oculus VR itd.), ova kompanija zadržava svoje mesto u svetu društvenih mreža, šireći svoj uticaj na mobilne tehnologije i virtuelnu realnost. Fejsbuk je u današnje vreme jedna od najpopularnijih društvenih mreža, a posedovanje naloga se podrazumeva i dok se nešto značajno ne promeni u svetu društvenih mreža, a to se realno može očekivati, Fejsbuk će držati apsolutni primat u komunikaciji među ljudima. Fejsbuk nudi brojne mogućnosti za unapređivanje čovekovog života kako na ličnom, tako i na poslovnom planu.
		\subsection{Twitter}
		\label{subsec:podnaslov2}
		Tviter(Twitter) je lansiran 2006. godine. Uprkos popularnosti, on polako gubi svoj uticaj povlačeći se pred Instagramom. Ova društvena mreža nikad neće imati efekat i masovnost kao Fejsbuk, ali zato može imati određeni uticaj. Tviter u Srbiji je pre svega mesto za takozvane tviteraše koji sebe smatraju „influenserima“. Kod nas je komunikacija na Tviteru manje bitna od komunikacije na Fejsbuku i Instagramu, ali je popularan kod tzv. društvene elite. Mnoge javne ličnosti, osobe visokog autoriteta i kredibiliteta, koriste ovu društvenu mrežu za sopstvenu promociju i nametanje svojih stavova. Tvitovi su posebno pogodni za brze i značajne objave koje poboljšavaju imidž samog tviteraša. Važno je napomenuti da je Tviter mnogo popularniji na svetskom nivou, nego u našem okruženju, gde mnogi prate poznate ličnosti i njihove kratke objave iz privatnog i javnog života. Prema svemu sudeći, iako nije popularan kao ostale društvene mreže, može se zaključiti da Tviter vlada iz senke, jer u mnogome utiče na stavove pojedinaca, naročito mladih.
		\subsection{Instagram}
		\label{subsec:podnaslov3}
		Instagram je veoma interesantna društvena mreža koja sve više liči na Fejsbuk. Specifičnost Instagrama je fokusiranost na fotografije, pri čemu je propratni tekst bitan, iako je on samo dodatak uz fotografiju. Ovu društvenu mrežu najviše prate mlade žene zainteresovane za modu, fitnes, kao i poznate ličnosti. Treba primetiti da se sa pojavom ove društvene mreže fokus sa komunikacije prebaca na estetiku i generalno vizuelizaciju ličnosti.
		\subsection{Snapchat}
		\label{subsec:podnaslov4}
		Snepčet (Snapchat) je od svih uticajnih društvenih mreža najmlađi, ali je trenutno veoma popularan. Specifičnost ove društvene mreže je što u startu nije ni imala ambicije da bude mreža, već je njen cilj bio kreiranje fotografija koje će nestati po viđenju. Interesantna početna ideja se pretvorila u multimilionsku kompaniju koja je najpopularnija među mlađim ljudima. Ukoliko se ciljna grupa odnosi na mlade, ova mreža je apsolutno neophodna, jer je u ovoj demografskoj grupi popularnija i od Fejsbuka i od Instagrama. Jedan od razloga za ovoliku popularnost Snepčeta kod mladih je činjenica što većina korisnika nija starija od trideset godina. U tome i jeste prednost ove mreže, jer možete biti sigurni da su vašu poruku videli samo mladi. Ako se tako posmatra stvar, Snepčet dobija prioritet uz Instagram, dok Fejsbuk ostaje na trećem mestu.
		\subsection{Zaključak}
		\label{subsec:podnaslov5}
		Iz pređašnjeg pregleda možemo izvesti zaključak da su društvene mreže kroz vreme dobile jedan potpuno drugi smisao. Fokus sa komunikacije usmeren je na celokupan život pojedinca. U to ubrajamo eksponiranje fizičkog izgleda, stavova ali i najdubljih emocija. Odjednom sve postaje transparentno i čovekova intima se stavlja u drugi plan.
		
		
		\section{Negativni uticaji društvenih mreža}	
		\label{sec:termini_i_citiranje}
		\subsection{Anksioznost}
		\label{subsec:podnaslov4}
		
		Strepnja, teskoba ili anksionznost je nejasan strah ili nemir bez očiglednog spoljnjeg povoda i uglavnom bez organizmičkih znakova koji uobičajeno prate strah proizašao iz racionalne, konkretne pretnje. Od straha se razlikuje zbog svoje neodređenosti, jer nije vezan za neki određeni objekat. U ovom neprijatnom osećanju dominira iščekivanje neke neodređene nesreće. Anksioznost je prisutnija kod žena. \\
		\textbf{Sindrom fantomske vibracije} se direktno vezuje za neke tipove anksioznosti. Nastaje tako što telefon u vašem džepu postaje poput dela tela zbog čega lako zaboravljamo da je tamo. Ogleda se u tome da nam se čini da mobilni vibrira u našem džepu što ukazuje na našu stalnu podsvesnu brigu ili želju za pristizanjem neke informacije(poruke, poziva ...). Postajemo opterećeni i napeti da toga nismo ni svesni. Sve to nam odvlači pažnju i skreće fokus. Stručnjaci upozoravaju na to da ovaj sindrom, pored anksioznosti, može dovesti do dugoročnih negativnih zdravstvenih posledica, kao što su glavobolja i poremećaji spavanja. Nedavna istraživanja Roberta Rosenbergera, filozofa i docenta na Tehnološkom institutu u Georgiji(Džordžiji), pokazala su da 90 odsto studenata ima ovaj sindrom.
		
		\subsection{Depresija}
		\label{subsec:depresija}
		
		
		Depresija predstavlja patološko neraspoloženje, odnosno tugu, koja svojom izraženošću, trajanjem i srazmerom prevazilazi uzroke. Skoro neizostavno je praćena gubitkom interesovanja i zadovoljstva, osećanjem krivice i bezvrednosti, niskim samopoštovanjem, pesimizmom, izmenom apetita i sna, ali i telesnim simptomima, poput iscrpljenosti, umora ili uznemirenosti. Samim tim, depresija negativno utiče na celokupan čovekov život, njegovo lično, porodično, društveno i radno funkcionisanje. Depresivne osobe ponekad potpuno gube volju za životom, odustaju i traže spas u samoubistvu.
		Brojne studije potvrđuju da se depresija, posebno manje teške forme, češće javlja u sredinama koje su na višem civilizacijskom nivou. Savremena društva su dinamična, brzo se menjaju, što zahteva stalno prilagođavanje pojedinca, čini ga usamljenim, nesigurnim  i otuđenim od sopstvenog „ja“. Pojedinac „dobrovoljno“ prihvata nametnuta pravila i sistem vrednosti da ne bi postao različit i odbačen. Ovakva frustracija iscrpljuje i zbunjuje čoveka, čini da se oseća teskobno, neshvaćeno i neispunjeno, što neminovno vodi ka depresivnom reagovanju. Društvo na brojne, formalne i neformalne načine šalje poruke kako treba živeti: kroz (pred)školski i pravosudni sistem, putem štampanih i radiodifuznih medija, ali posebno posredstvom interneta i društvenih mreža. Propagira se jednoličnost i uniformnost. Negativno poređenje sebe s drugima, čije su fotografije estetski idealne ili im statusi govore da žive srećno, često putuju, provode se, neminovno će uticati na gubitak samopouzdanja i nezadovoljstvo sopstvenim životom. Istovremeno će „savršeni portreti“ drugih buditi potrebu da i sami stvorimo idealnu (lažnu) sliku o sebi i pokažemo je svetu. Osećaji duboke depresije otkriveni su kod osoba koje su većinu vremena provodile na društvenim mrežama i gradile veb imidž. Takođe, za depresiju se vezuje loša socijalna interakcija koju imaju osobe koje većinu vremena provode na društvenim mrežama.
		
		\subsection{Stres}
		
		Činjenica je da u današnje vreme teško možemo da se potpuno oslobodimo društvenih mreža. Pregršt bitnih i manje bitnih informacija okupira celokupnu populaciju, počev od najmađih. Za stres se može reći da se ispoljava u paraleli sa anksioznošću. Kada se kao stariji susrećemo sa mnogo novih informacija, možemo se fokusirati na one nama bitnije, dok deca nisu u stanju racionalno da rasuđuju i iz tog razloga gube percepciju koja to informacija je bitna, a koja samo nepotrebno optereti dečiji mozak. Deca tako gube interesovanje, gube empatiju i strpljenje. Kod srednjoškolaca je pokazano da povećani nivo stresa zbog društvenih mreža ima jasan uticaj na smanjenu fokusiranost u nastavi.\\
		\textbf{Akulturativni stres} je tip stresa koji je povezan sa imigrantskom ili etničkom manjinom i prolaskom kroz proces akulturacije tj. prilagođavanja lokalnoj kulturi. Ovaj stres izvazvan je raznim faktorima kao što su status porodice, društveni položaj i slično. Danas, kad se vrednuje eksponiranje u virtuelnom svetu, među adolescentima imamo visoku zastupljenost akulturativnog stresa. Deca prolaze kroz svakodnevni stres u nadi da će uspeti da se prilagode i postanu prihvaćeni od strane godišnjaka, time što će biti u toku sa dešavanjima na društvenim mrežama, time što će ostvariti određenji broj lajkova, privući pažnju ili izgraditi neki opšte prihvaćen imidž i sl.
		
		\subsection{Zavisnost}
		
		Većina ljudi misli da je zavisnost vezana samo za upotrebu hemijskih sredstava kao što su alkohol, nikotin, kokain i heroin, ali stručnjaci veruju da sve što stimuliše ljudska bića i čini da se osećaju energično može izazvati zavisnost. Lečenje zavisnosti nije ograničeno samo na terapiji lekovima. Kad god navika postane prisila, kao što je kockanje, droga, alkohol ili čak igranje kompjuterskih igara, ćaskanje, surfovanje po mrežama, i to se može posmatrati kao zavisnost. Ono što ljude čini zavisnim je prijatno iskustvo u obavljanju te radnje, kao što se zavisnici osećaju kada koriste hemijske droge. U korišćenju interneta se ona ispoljava kroz osećanje emocionalne sigurnosti uz internet.\\
		Ukoliko se u odgovorima na pitanja navede najmanje pet pozitivnih odgovora može se prepoznati osoba zavisna od interneta:
		\begin{itemize}
			\item
			Da li se osećate preokupirani internetom?
			\item
			Da li osećate potrebu da sve duže koristite internet da biste postigli zadovoljstvo?
			\item
			Da li ste više puta pokušavali da se kontrolišete, smanjite ili zaustavite upotrebu interneta?
			\item
			Da li se osećate nemirno, ćudljivo, depresivno ili razdražljivo kada pokušate da smanjite ili zaustavite upotrebu interneta?
			\item
			Da li ostajete na vezi duže nego što je prvobitno planirano?
			\item
			Da li ste ugrozili ili rizikovali gubitak socijalnih odnosa, posla, obrazovnih ili karijernih mogućnosti zbog interneta?
			\item
			Da li ste lagali članove porodice ili druge da bi prikrili stepen uključenosti na internet?
			\item Da li koristite internet kao način da pobegnete od problema ili olakšate disfonično raspoloženje(npr. osećanja krivice, anksioznosti i depresije)?
		\end{itemize}
		Aplikacije i veb stranice društvenih medija imaju isti efekat na mozak kao igranje slot mašina. Pošto ne znate sadržaj koji ćete videti dok ne otvorite aplikaciju, spontani rezultati upravo izazivaju osećaj „nagrade“ oslobađanjem dopamina – iste supstance koja je povezana sa drugim prijatnim aktivnosti kao što je npr. uživanje u hrani ili masaži.\\\\
		Ono što direktno ukazuje da ste potencijalni zavisnik je i \textbf{FOMO} efekat. On se opisuje kao osećaj da postoje bolje stvari koje bismo mogli da radimo u ovom trenutku umesto onoga što zaista radimo. To je osećaj da propuštamo nešto od velike važnosti(socijalna interakcija ili novo iskustvo) što drugi doživljavaju. Stručnjaci navode da do ove pojave može doći usled nedovoljno zadovoljenih psiholoških, a pre svega socijalnih potreba. Kod adolescenata su često u pitanju potreba za pripadanjem i potreba za popularnošću. FOMO postoji mnogo duže od društvenih medija, čini se da sajtovi kao što su Facebook i Instagram pogoršaju osećaj teskobe pojedinca zbog pomisli da se drugi više zabavljaju ili žive bolje. Ideja da propuštate neke stvari može uticati na vaše samopoštovanje, izazvati anksioznost i podstaći još veću upotrebu društvenih medija. FOMO vas može primorati da podignite telefon svakih nekoliko minuta da proverite da li postoji ažuriranje ili kompulzivno odgovarate na svako upozorenje - čak i ako to znači da rizikujete dok vozite, da ne spavate noću i sl. Studija sprovedena na uzorku studenata jednog američkog univerziteta, pokazuje da je FoMO u pozitivnoj korelaciji sa vremenom provedenim na društvenim mrežama. Jedan od primera za FoMo efekat je da pojedinci osećaju ogroman nivo uznemirenosti  ili napetosti ako određeno vreme ne ulaze na Instagram, pa moraju da pogleduju svaku objavu koju su propustili.
		
		\subsection{Delikvencija}
		
		Sredstva masovnih komunikacija se u neprestanoj trci za senzacionalizmom ne
		libe da objavljuju monstruozne zločine, iako su upoznati sa činjenicom da u velikoj meri utiču na razvijanje svesti pre svega kod maloletnika, na sadržaj njihovih misli i u krajnjem slučaju na njihova kriminalna dela prezentovanjem sadržaja koji su prepuni zločina, kriminala i krvi. Prisutnosto nasilja u stampanim medijima, na televiziji, internetu kao i u kompjuterskim igricama, nepobitna je činjenica. Iz tih i mnogih drugih razloga ustanovljeno je da se empatija kod mladih smanjuje. Brzo širenje propagande i negativnog uticaja najviše pogađa mlađu populaciju. Zato se postavlja pitanje kada, gde, šta i u kojoj meri se plasira. Jedan od direktnih primera maloletne delikvencije jesu i tuče najavljivane preko neke društvene mreže, takvi slučajevi su zabeleženi i kod nas u više navrata. Još jedan drastičan primer je prodaja narkotika preko društvenih mreža koja je zabeležena od strane maloletnika.
		
		\subsection{Samopovređivanje i suicidne misli}
		
		Razna istraživanja su pokazala da vreme provedeno na društevnim mrežama statistički zanačajno korelira sa ponašanjem samopovređivanja. Isto tako treba naglasiti da data istrazivanja nisu ustanovila direktnu vezu. Svedoci smo sve češćeg nasilja i otvorene diskriminacije na društvenim mrežama. U delu \eqref{subsec:depresija} smo pričali o uticaju depresije na pojedinca, ako se ona ne leči postoji bojaznost da može dovesti do tragičnih posledica.
		
		\section{Pozitivni uticaji društvenih mreža}	
		Ljudska bića su društvena, oslanjaju se na zajednicu. Potrebno nam je društvo drugih da bismo napredovali u životu, a snaga naših veza ima ogroman uticaj na naše mentalno zdravlje i sreću. Biti društveno povezan sa drugima može ublažiti stres, anksioznost i depresiju, povećati samopoštovanje, pružiti udobnost i radost, sprečiti usamljenost, pa čak i dodati godine vašem životu. Sa druge strane, nedostatak jakih društvenih veza može predstavljati značajan rizik za vaše mentalno i emocionalno zdravlje. Društvene mreže nam omogućavaju da:
		\begin{itemize}
		\item	Komuniciramo i budemo u toku sa porodicom i prijateljima širim sveta.
		\item Pronađemo nove prijatelje i zajednice.
		\item Umrežimo se sa drugim ljudima koji dele slične interese ili ambicije.
		\item Pridružimo se ili promovišemo vredne ciljeve i podižemo svest o važnim pitanjima.
		\item Potražimo ili ponudimo emocionalnu podršku tokom teških vremena.
		\item Pronađemo vitalnu društvenu vezu ako živimo u udaljenom području, ili imamo ograničenu nezavisnost, društvenu anksioznost ili smo deo marginalizovane grupe.
		\item Pronađemo izlaz za svoju kreativnost i samoizražavanje.
		\item Otkrijemo izvore vrednih informacija i učenja.
		\item Lakše dolazimo do posla.
		
		\end{itemize}
	
		\section{Dve studije}	
		Postoje dve naučne studije koje su izučavale uticaj prekomerne upotrebe društvenih mreža na zdravlje ljudi. Prva studija propagira mišljenje kako preterano korišćenje društvenih mreža ima uticaj na mentalno zdravlje ljudi, dok druga pokušava da pokaže nezavisnost među ovim pojmovima.\\
		\subsection{Prva studija}
		Prva studija se zasniva na naučnom radu dvojice Indijaca "Effects of Social Media on Mental Health: A Review" \cite{prva} (Hilal Bashir i Shabir Ahmad Bhat) koji su izneli stavove potkrepljene velikim brojem naučnih istomišljenika koji govore o pozitivnoj korelaciji između korišćenja društvenih mreža i mentalnog zdravlja. Anksioznost je jedan od najprimećenijih poremećaja kod omladine. Utvrđeno je testiranjem da 45 procenata osoba u Britaniji oseća nemir kada nema mogućnost pristupa društvenim mrežama. Takođe, Drouin \cite{Drouin} i Rothberg \cite{Rothberg} su pokazili kako omladina ima sindrom fantomske vibracije. Kod srednjoškolaca je pokazano \cite{bashir} da povećani nivo stresa zbog društvenih mreža ima jasan uticaj na smanjenu fokusiranost u nastavi.  U istraživanjima koja su obavljena 2012. (Pantic \cite{panic}) i 2013. (Rosen \cite{rosen}) godine došlo se do zaključka da su vreme provedeno na Facebook-u među adolescentima i depresija u pozitivnoj korelaciji. Osećaji duboke depresije otkriveni su kod osoba koje su većinu vremena provodile na društvenim mrežama i gradile veb imidž. Takođe, za depresiju se vezuje loša socijalna interakcija koju imaju osobe koje većinu vremena provode na društvenim mrežama.\\
		\subsection{Druga studija}
		\label{dr}
		U ovoj studiji je prikazano osmogodišnje istraživanje obavljeno na 500 adolescenata čiji je razvoj praćen od 13-e do 21-e godine. Jednom godišnje učesnici su popunjavali ankete i praćen je njihov indvidualni razvoj kao i generalni razvoj cele grupe. Koliko je teško diskutovati na ovu temu, možemo videti u sledećim naučnim ispitivanjima. Na osnovu hipoteze o raseljavanju (Lin, 1993 \cite{lin}), vreme provedeno na društvenih mrežama može da iseli druge aktivnosti koje pozitivno utiču na mentalno zdravlje kao što su spavanje i vreme provedeno sa prijateljima. Suprotno tome, teorija korišćenja i zadovoljenja (Katz, Blumler i Gurevitch, 1974 \cite{katz}) govori da se kod lošeg mentalnog zdravlja može predvideti češće korišćenje društvenih mreža. Odnosno, osobe koje se osećaju depresivno beže u virtuelni svet kako bi se tamo spojile sa drugim ljudima. Na osnovu izvršenih testiranja, jasno se uočilo povećanje vremena provedenog na društšvenim mrežama. Sa 13 godina, to vreme je iznosilo oko 31-60 minuta, dok u mlađoj odrasloj dobi i preko 2h dnevno. Takođe, problemi mentalnog zdravlja rasli su vremenom, ali su bili u granicama normalnih. Takođe, kod devojčica je primećena veća korelacija između vremena provedenog na društvenim mrežama i problema mentalnog zdravlja. Rezultati na nivou grupe ispitanih pokazali su porast u korelaciji korišćenja društvenih mreža i simpota depresije u periodu od 13-e do 18-e godine i blagi pad korelacije nakon toga. Međutim, na individualnom nivou, nije se moglo zaključiti da će osoba imati veći nivo depresivnosti ako poveća vreme korišćenja društvenih mreža. Odnosno, iako u širem gledanju postoji korelacija među promenljivima, na individualnom nivou se ne može zaključiti stopa povezanosti i predviđanja. Jedini izuzetak je kod 16-ogodišnjih dečaka sa povećanim simptomima depresije. Slično kao kod depresije, simptomi anskioznosti i korišćenje društvenih mreža imaju
		povezanost samo na nivou cele grupe. U ovom istraživanju zaključak na osnovu rada američkih naučnika ukazuje na potpuno drugačiji zaključak ako problem posmatramo iz ugla pojedinca, nego kada ga posmatramo iz ugla grupe.\\
		Kako je moguće da postoje dve strane istine? Problem drušvenih mreža i mentalnog zdravlja nije nimalo lak problem za rešavanje. Zato je veoma bitno jasno postaviti činjenice na početku i odrediti uzročno-posledične veze na kraju. Rezultati koji su dobijeni u ovim radovima, poklapaju se kada problem posmatramo šire. Naime, u američkom istraživanju \eqref{dr}, kada su se obradili svi dobijeni rezultati, postoji pozitivna korelacija između vremena provedenog na društvenim mrežama i problema mentalnog zdravlja. Verovatno su i naučnici navedeni u literaturi zaključivali o korelaciji na osnovu rezultata grupe. Njihovo mišljenje je da postoji negativan uticaj društvenih mreža na mentalno zdravlje. Međutim, na individualnom nivou ne može se govoriti o ovom uticaju. Iako je ovaj uzorak zaista raznovrstan, opet je veoma mali da bi se donosili globalni zaključci. 
		
		\section{Potencijalna rešenja}	
		 Postavljaju se pitanja ima li rešenja za prethodno navedene probleme i najčešće ćemo naići na potvrdan odgovor. Naime, treba uspostaviti balans i imati meru. Mnogima od nas to ne ide tako lako od ruke. Tekovine savremenog društva zapostavljaju težnju čoveka da ostvari sopstveni mir. Ono što možemo u svakom trenutku da sprovedemo jesu neke od narednih mera:
		 \begin{itemize}
		 	\item Okrenite se prirodi.
		 	\item Bavite se sportom ili nađite hobi koji nema direktan kontakt sa internetom.
		 	\item Postavite jasne granice.
		 	\item Zatražite pomoć. Prvi korak ka boljitku je prihvatanje problema.
		 	\item Budite prijatelj vašoj deci, kontrolišite ih da bi reagovali dok nije kasno.
		 	\item Ne dozvolite da društvene mreže koriste vas, vi koristite njih.
		 			 	
		 \end{itemize}  
		
	
	
		\addcontentsline{toc}{section}{Literatura}
		\appendix
		
		\iffalse
		\bibliography{seminarski} 
		\bibliographystyle{plain}
		\fi
		
		\begin{thebibliography}{9}
			
			\bibitem{prva} H. B. Shabir Bhat, Effects of Social Media on Mental Health: A Review. The International Journal of Indian Psychology, January 2016.
			\bibitem{Drouin}  K. D. H. . M. D. A. Drouin, M., Phantom vibrations among undergraduates: Prevalence and associated psychological characteristics. Computers in Human Behavior, 2012.
			\bibitem{Rothberg}A. A. H. J. S. M. P. . V. P. Rothberg, M. B., Phantom vibration syndrome among medical staff: A cross sectional survey. British Medical Journal, 2010.
			\bibitem{bashir} L. Kaur, R. Bashir, Impact of stress on mental health of students: Reasons and Interventions. International Journal of Education, 2016.
			\bibitem{panic}I. Pantic, Association between online social networking and depression in high school students: Behavioral physiology viewpoint. Psychiatria Danubina, 2012.
			\bibitem{rosen}W. K. R. S. C. L. . C. N. Rosen, L.D., The link between clinical symptoms of psychiatric disorders and technology use, attitudes and anxiety. Computers in Human Behavior, 2013.
			\bibitem{lin} C. A. Lin, Exploring the role of VCR use in the emerging home entertainment culture. Journalism Quarterly, 1993.
			\bibitem{katz} B. J. G. . G. M. Katz, E., Utilization of mass communication by the individual. The uses of mass communications: Current perspectives on gratifications research, 1974.
			
			
			
		\end{thebibliography}
		
		
		
		
	\end{document}
