%-----------------------------------------------------------------------------
%   PACKAGES
%-----------------------------------------------------------------------------
\documentclass[12pt, a4paper, twocolumn]{article}
\usepackage[utf8]{inputenc}
\usepackage[T2A]{fontenc}
\usepackage[serbian]{babel}
\usepackage[margin=1in]{geometry} 
\usepackage{graphicx}
\usepackage{pgfplots}
\usepackage[backend=biber,
            natbib=true,
            url=false,
            doi=true,
            eprint=false
]{biblatex}

\pgfplotsset{compat=1.16}


%-----------------------------------------------------------------------------
%   TITLE PAGE
%-----------------------------------------------------------------------------
\title{Capture The Flag (CTF) kao uvod u računarsku bezbednost\\Radna verzija(nije za čitanje, u mnogim delovima nećete razumeti šta je pisac hteo da kaže)\\}
\author{Andrija Urošević\\Univerzitet u Beogradu\\Matematički fakultet}
\date{April, 2022}


%-----------------------------------------------------------------------------
%   BIBLIOGRAPHY
%-----------------------------------------------------------------------------
\addbibresource{ctf_main.bib}


%-----------------------------------------------------------------------------
%   DOCUMENT
%-----------------------------------------------------------------------------
\begin{document}

\maketitle

%-----------------------------------------------------------------------------
%   ABSTRACT
%-----------------------------------------------------------------------------
\begin{abstract}
    \textbf{Ključne reči}: \emph{Capture The Flag}, \emph{CTF}, 
    \emph{računarska bezbednost}, \emph{internet bezbednost}, \emph{učenje}, 
    \emph{edukacija}.
\end{abstract}

%-----------------------------------------------------------------------------
\section{Uvod}
%-----------------------------------------------------------------------------

%-----------------------------------------------------------------------------
\section{CTF takmičenja}
%-----------------------------------------------------------------------------

Termin \emph{Capture The Flag} (CTF) se originalno odnosi na igru između dva
tima. Svaki tim ima zadatak da sačuva svoju fizičku zastavicu (\emph{flag}), 
dok u isto vreme pokušava da osvoji zastavicu (\emph{flag}) drugog tima.
Od 90-tih, format CTF-a se premešta na računare i Internet. 

\emph{Capture The Flag} (CTF) je takmičenje u oblasti računarske bezbednosti.
Cilj takmičenja je pronaći \emph{flag}-ove u nekom okruženju. Okruženje može
biti različitog opsega i formata, ali generalno pronađen \emph{flag} je dokaz
rešenog zadatka (npr.\ pristupljeno je skrivenim podacima ili bazi). Okruženje 
može  biti jedan veb domen ili može biti mreža računara na većoj skali.

\emph{Flag} je tipično neki string karaktera, čiju specifikaciju daju
organizatori takmičenja. Specifikacija mora biti jasna svim učesnicima kako 
bi znali da su uspešno pronašli \emph{flag} i rešili zadatak. \emph{Flag}
obično ima neki prefiks koji učesnicima ukazuje da su zapravo pronašli 
\emph{flag}. Na primer, \emph{flag} može imati prefiks oblika \texttt{FLAG},
a sam \emph{flag} može izgledati kao \texttt{FLAG\{K73BSSxY3nFc1oAs9WwG\}}.
Prostor \emph{flag}-ova mora biti dovoljno velik, kako će ta činjenica 
sprečiti takmičare da koriste metod grube sile za pronalaženje \emph{flag}-a. 
Od takmičara se očekuje da pronađu \emph{flag} u datom okruženju.

Kada takmičar pronađe \emph{flag}, on ga šalje sistemu za verifikaciju, 
i ukoliko je \emph{flag} ispravan takmičar dobija poene za odgovarajući 
zadatak koji je rešio. Ovaj sistem je obično realizovan preko veb aplikacije, 
gde se takmičari mogu prijaviti, poslati pronađene \emph{flag}-ove, i uživo 
pratiti trenutne rezultate.

Nakon uspešno završenog takmičenja, javlja se potreba kod takmičara da podele
svoja genijalna rešenja sa ostalim učesnicima. Za osvojene \emph{flag}-ove
takmičari pišu iscrpe korake kako doći do tog \emph{flag}-a. Dokumenti koji 
nastaju nazivaju se \emph{writeup}-ovi. Preko \emph{writeup}-ova učesnici mogu 
da uporede svoja rešenja ili pronađu određeno rešenje neuspešnog pokušaja 
pri rešavanju. U ovom procesu učesnici potkrepljuju svoja znanja, 
i stiču nove tehnike.

Postoje 3 glavne vrste CTF takmičenja: \emph{Jeopardy}, \emph{Attack-Defence},
i \emph{Mixed}. \cite{ctf_time} U \emph{Jeopardy} CTF-u, takmičari dobiju 
unapred zadatke zadatke, koji su zadati u statičkom sistemu.
\emph{Attack-Defence} stil podrazumeva borbu između više timova. Svaki tim
ima sopstvenu mrežu koja ima ranjive servise. Cilj je ``zakrpiti'' ranjive
servise, i u isto vreme eksploatisati druge timove. Za svaku uspešnu odbranu 
od napada, ili uspešan napad timovi dobijaju \emph{flag} sa određenim brojem 
poena. \emph{Mixed} CTF može biti različitog formata, ali kao što ime sugeriše 
predstavlja mešavinu prethodna dva stila.

Popularnost CTF takmičenja raste iz godinu u godinu. \texttt{CTFTime} u
svojoj arhivi ima preko 300 javno dostupnih CTF takmičenja za 2021.\ godinu,
dok za 2016.\ godinu ima preko 100 javno dostupnih CTF takmičenja.
\cite{ctf_time} Jedno od prvih i najpoznatijih CTF takmičenja je DEFCON CTF
koji se svake godine održava na DEFCON konferenciji o računarskoj bezbednosti. 
\cite{ctf_defcon} Pored DEFCON CTF-a postoje i mnoga druga CTF takmičenja kao 
što su UCSB iCTF, Mozilla CTF, Facebook CTF, Google CTF, PHD CTF, RuCTFe, 
Hack.lu CTF, SECUINSIDE CTF, rwth CTF, CSAW CTF, PICO CTF. \cite{ctf_rank} 
Jedan interesantan CTF, koji se održava u Srbiji svake godine, u okviru 
DESCON hakatona je DESCON CTF. \cite{ctf_descon} DESCON CTF je namenjen 
za početnike.

%-----------------------------------------------------------------------------
\section{Znanja i veštine koje se stiču kroz CTF}
%-----------------------------------------------------------------------------

Treniranje profesionalaca u oblasti računarske bezbednosti zahteva puno 
vremena i novca, ali pruža jedno veoma održivo globalno rešenje. Mnoge 
obrazovne institucije, društva informatičara, državne organizacije, i privatne
kompanije su svesne toga te konstantno uvode nove studijske programe, 
i kurseve. Jedan od tih studijskih programa je CSEC2017. \cite{ctf_csec}

Pored formalnog obrazovanja, povećava se popularnost neformalnih metoda.
CTF predstavlja jednu takvu metodu gde učesnici poboljšavaju svoje znanje u
oblasti računarske bezbednosti kroz razne zadatke. Kako CTF zadaci često 
poseduju takmičarske elemente i elemente igre, oni su neformalnog karaktera
i teško je odrediti njihovu vezu sa formalnim metodama.

Postavlja se pitanje o tome kako i koliko su povezani formalni metodi,
u vidu studijskih programa, sa neformalnim metodama poput CTF-a. U daljem 
tekstu su opisane oblasti znanja koje definiše CSEC2017, nakon čega sledi 
studija o distribuciji tih oblasti znanja u CTF zadacima.

%-----------------------------------------------------------------------------
\subsection{CSEC2017 oblasti znanja}
%-----------------------------------------------------------------------------

CSEC2017 definiše osam oblasti znanja u računarskoj bezbednosti.
\begin{enumerate}
    \item \emph{Bezbednost podataka} sadrži kriptografiju, forenziku, 
        integritet podataka, i autentifikaciju.
    \item \emph{Bezbednost softvera} se fokusira na bezbednost u 
        programiranju, testiranju, i druge aspekte razvoja softvera.
    \item \emph{Bezbednost komponenti} se odnosi na bezbednosti komponenti 
        koje se integrišu u veće sisteme, što uključuje njihov dizajn i 
        obrnuto inženjerstvo.
    \item \emph{Bezbednost konekcije} podrazumeva mrežne servise, odbrane, i 
        napade.
    \item \emph{Bezbednost sistema} sadrži kontrolu pristupa, i etičko 
        hakovanje (eng.\ \emph{pen testing}).
    \item \emph{Bezbednost ljudi} se odnosi na čuvanju identiteta, podataka, 
        i privatnosti. Sadrži socijalno inženjerstvo i svesnost o računarskoj 
        bezbednosti.
    \item \emph{Organizaciona bezbednost} se fokusira na menadžment rizika, 
        bezbednosne polise, i upravljanje incidentima na nivou organizacije.
    \item \emph{Društvena bezbednost} se bavi računarskom bezbednošću 
        na nacionalnom ili globalnom nivou.
\end{enumerate}

%-----------------------------------------------------------------------------
\subsection{Distribucija oblasti znanja u CTF zadacima}
%-----------------------------------------------------------------------------

Švábenský i dr.\cite{ctf_skills} ispitivali su distribuciju oblasti znanja u 
CTF zadacima. Ispitivanje je vršeno nad podacima, koji čine $5963$ 
\emph{writeup}-ova. Ovi podaci su preuzeti sa \url{CTFTime.org} koji u svojoj 
bazi, između ostalog, čuva i \emph{writeup}-ove raznih zadataka sa CTF 
takmičenja.\cite{ctf_time}

\textbf{Metod.} Prva faza je izdvajanje ključnih reči iz CSEC2017, koje 
određuju svako od znanja. Druga faza je preuzimanje i parsiranje 
\emph{writeup}-ova sa \url{CTFTime.org}. Treća faza predstavlja analizu 
\emph{writeup}-ova, tj.\ prebrojavanje instanci ključnih reči u svakom 
\emph{writeup}-u. Sledeća, četvrta faza predstavlja normalizaciju 
broja instanci ključnih reči. Poslednja, peta faza se sastoji u dodeljivanju 
\emph{writeup}-ova odgovarajućoj oblasti znanja.\cite{ctf_skills}

\textbf{Rezultati.} Najzastupljenija oblast znanja za \emph{Jeopardy} stil
CTF-a je \emph{bezbednost podataka}, dok \emph{bezbednost konekcije} i 
\emph{bezbednost sistema} zauzimaju drugo i treće mesto, respektivno. 
\emph{Bezbednost podataka} uključuje kriptografiju, i autentifikaciju, 
što opravdava prvo mesto zbog same prirode zadataka iz tih oblasti. Naime,
takvi zadaci su laki za dizajn i proveru. Za \emph{Attack-Defence} stil CTF-a
dobija se da je \emph{bezbednost konekcija} na prvom mestu. Ovaj rezultat ima
smisla kako timovi konstantno vrše napade na druge timove. Ostali rezultati 
se nalaze na slici~\ref{fig:ctf_ka}.\cite{ctf_skills} Interesantno je to da
dobijeni rezultati odgovaraju rezultatima o distribuciji oblasti znanja na 
master studijskim programima iz kurseva o računarskoj 
bezbednosti.\cite{oth_ka, ctf_skills}

\begin{figure}
    \begin{center}
        \begin{tikzpicture}[scale=0.6]
            \begin{axis}[
                    y=1cm, 
                    xbar, 
                    title={Distribucija oblasti znanja u CTF-u}, 
                    symbolic y coords={
                        Društvo, 
                        Organizacija, 
                        Ljudi, 
                        Sistemi, 
                        Konekcije, 
                        Komponente, 
                        Softver, 
                        Podaci
                    },
                    legend pos = south east, 
                    nodes near coords, 
                    xmax=50
                ]
                \addplot+ coordinates {
                    (2.96,Društvo) 
                    (9.86,Organizacija) 
                    (8.23,Ljudi) 
                    (12.72,Sistemi) 
                    (19.66,Konekcije) 
                    (8.94,Komponente) 
                    (10.02,Softver) 
                    (27.61,Podaci)
                }; 
                \addlegendentry{Jeopardy}
                \addplot+ coordinates {
                    (2.17,Društvo) 
                    (11.38,Organizacija) 
                    (9.18,Ljudi) 
                    (10.96,Sistemi) 
                    (32.68,Konekcije) 
                    (2.08,Komponente) 
                    (14.72,Softver) 
                    (16.83,Podaci)
                }; 
                \addlegendentry{Attack-Defence}
            \end{axis}
        \end{tikzpicture}
    \end{center}
    \caption{
        Distribucija oblasti znanja u $15 879$ \emph{jeopardy} i 
        $86$ \emph{attack-defense} \emph{writeup}-ova.\cite{ctf_skills}
    }\label{fig:ctf_ka}
\end{figure}

\textbf{Diskusija.} Glavno ograničenje ove analize je mali skupu podataka 
nad kojima je analiza vršena, zajedno sa odbacivanjem polovine skupa podataka 
zbog neuspešnog parsiranja.\cite{ctf_skills} Postavlja se, takođe, pitanje o 
pouzdanosti samih \emph{writeup}-ova, tj.\ o njihovoj povezanosti sa samim
CTF zadatkom. Pod pretpostavkom da \emph{writeup}-ove pišu entuzijasti i sami
dizajneri CTF zadataka, možemo pretpostaviti njihovu pouzdanost.

%-----------------------------------------------------------------------------
\section{Problemi u CTF modelu}
%-----------------------------------------------------------------------------

Cilj CTF takmičenja je okupiti ljude koji se bave računarskom bezbednošću
različitog nivoa znanja i veština, kako bi oni mogli međusobno da dele
informacije, pored samog takmičarskog elementa. Čitava zajednica poštuje
CTF kao platformu za učenje novih veština, ali nekolicina članova direktno
priča o problemima sa kojima se susreću organizatori i takmičari.

CTF se vrti oko računarske bezbednosti, ali je u suštini slobodan za bilo koje
druge oblasti. Zbog toga CTF zadatak može testirati dosta nepoznate oblasti 
znanja, koje se samo delimično mogu naći u literaturi, dokumentaciji, ili čuti 
na nekom kursu ili radionici. Dobra strana slobode koju CTF-ovi pružaju je u 
tome što otvara nove teme i uzdiže bezbednost na veći nivo. 

Sama struktura CTF zadataka se svodi na ``sve ili nista'', tj.\ nije moguće
parcijalno rešiti neki zadatak. To otvara mogućnosti za dalje istraživanje, i
zajedno sa ograničenim vremenom forsira takmičare da budu efikasniji. Čak i
kada takmičar nije na pravom putu on može naučiti o nekoj oblasti. Ovo stvara
okruženje u kome najuporniji pobeđuju, bez obzira na njihov tehnički nivo.
Veruje se da ce uspešni takmičari biti oni koji se prepuste učenju i kulturi
CTF-ova.

Pored svih uspešnih stvari koje donosi CTF kao platforma za učenje i proveru
znanja, postoje problemi koji narušavaju ovaj model. Jedan od najvećih
probleme je uvođenje novih ljudi u takmičenja.\cite{ctf_chung} Novi takmičari
počinju sa takmičenjem po preporuci. Oni veoma retko nastavljaju sa igranjem
CTF-a ukoliko ne mogu da reše ni jedan zadatak. Ovo može biti veoma 
frustrirajuće, te mnogi odustaju od takmičenja.

Sledeće podsekcije sadrže neke probleme koje su uočili Čang i Koen 
\cite{ctf_chung} u višegodišnjoj organizacije CTF takmičenja.

%-----------------------------------------------------------------------------
\subsection{Težina igre}
%-----------------------------------------------------------------------------

CTF igre nisu jednostavne za igranje, tj.\ teško je ući u sam zadatak, čak i
za veterane, a pogotovu za nove takmičare. CTF zadatak objašnjava veoma
malo o teme kako ga rešiti, već prepušta igraču da sam smisli optimalan put
do rešenja, kao što je to u igrama poput šaha ili igre go. Baš na ovom aspektu
se dobija na mogućnosti izučavanja raznih oblasti. CTF podrazumeva da će
njegovi takmičari biti eksperti u svojoj oblasti, te iz godine u godinu
zajednica raste, a sami zadaci postaju sve teži. Ovo stvara veoma veliki 
pritisak na novog takmičara. Zbog toga je veoma teško početi sa takmičenjem,
čak i za one sa dobrim tehničkim predznanjem.

%-----------------------------------------------------------------------------
\subsection{
    Relacija izmedju dizajna zadataka i njegove uspešnosti pri rešavanju
}
%-----------------------------------------------------------------------------

Uspešan CTF događaj je onaj koji ima balans između učenja i provere znanja.
Zbog toga, organizatori imaju veoma težak posao da osmisle dovoljno teške 
zadatke iz kojih će takmičari steći znanje. Sa druge strane zadaci ne
smeju biti previše teški, jer se takmičari mogu zaglaviti na tom zadatku, i
pri tome se stvara rizik o izgubljenom vremenu čitave zajednice. Naime,
takmičari neće pogledati ostale zadatke, i samim tim neće rešiti neki lakši 
zadatak i steći priliku za učenje nove oblasti. Zbog toga dobri zadaci 
predstavljaju one zadatke koji navode do sopstvenog rešenja. Jedna metrika 
koja meri težinu zadatka može biti broj uspešnih rešenja. Lak zadatak će imati 
veliki broj uspešnih rešenja, dok će težak zadatak imati mali broj uspešnih 
rešenja.

Svakom zadatku organizatori dodeljuju poene, koje će takmičar dobiti pri
uspešnom rešavanju tog zadatka. Poeni su u direktnom odnosu sa težinom 
zadatka, pa će tako teški zadaci imati puno poena, dok će laki zadaci imati 
malo poena. Kako postoje mnogi CTF događaji, koji ciljaju na određene nivoe 
znanja, poeni predstavljaju lokalnu skalu težine zadataka tog CTF događaja.

Često organizatori pokušavaju da modifikuju zadatke tako da oni postanu teži
dodavanjem veštačkih konstrukcija, koji čine zadatak ``upekljanijim'' nego
što on zapravo jeste. Ovakve stvari su dizajnirane da namerno frustriraju
takmičare, i veruje se da će mali broj takmičara uspeti da ih reši.
Ove tehnike otežavanja dovode do toga da zadatak ostane nerešen ili rešen
od strane malog broja takmičara, jer zapravo uključuje faktor sreće. 

Još jedan način otežavanja zadatka je integrisanje grube sile u samo rešenje
zadatka. Gruba sila ne predstavlja dobar način savladavanja novih veština,
već samo dovodi do gubitka vremena pri rešavanju na CTF događaju gde je 
vreme veoma bitan faktor.

CTF takmičenja pored uobičajenih zadataka uključuju i veoma lake zadatke.
Laki zadaci podužu entuzijazam takmičara, i vagaju između igre i takmičenja. 
Oni su tu radi uživanja, jer će ih svaki takmičar rešiti bezmalo truda.

%-----------------------------------------------------------------------------
\subsection{Dokaz o kvalitetu}
%-----------------------------------------------------------------------------

Veterani su vešti u samom početku rešavanja zadatka, ali problem nastaje kod 
novajlija. Naime, novajlije treba uvesti u novu oblast, kroz uvod u zadatak. 
Ne treba takmičare u potpunosti navodit do rešenja, jer tu onda dolazi do 
slepog ispunjavanja uslova i gubi se na istraživanju i učenju. Kako bi ovaj 
problem bio smanjen, najbolje je imati tim ljudi koji su na istom nivou kao i 
takmičari. Oni znaju na koji način takmičari razmišljaju i mogu davati 
direktne povratne informacije o pravcu navođenja. 

Nakon što je zadatak napravljen ulazi se u fazu dokaza o kvalitetu. Za
svaki CTF događaj ova faza je drugačija. Jedan primer može biti u tome
da organizatori između sebe dele zadatke i pokušavaju da reše tuđi zadatak.
Organizatori koji rešavaju zadatke direktno mogu da revidiraju zadatak, i
da daju povratne informacije o tome kako ga popraviti. Onda započinje novi
krug razvoja zadataka, koja se sastoji u poboljšavanju zadataka.

Mnogi CTF događaji zanemaruju ovu fazu, ili joj ne pridaju veliki značaj.
To dovodi do nerešivih zadataka, neadekvatno konfigurisana infrastrukture,
i zadataka čiji poeni ne reflektuju njihovu težinu.

%-----------------------------------------------------------------------------
\subsection{Poeni i njihov obrnuti efekat na takmičare i organizatore}
%-----------------------------------------------------------------------------

Dolazimo do interesantnog zapažanja nakon što takmičari dobiju zadatke.
Naime, takmičari će veštački odrediti težinu zadatka, pre čitanja uvoda u 
zadatka, samo na osnovu njegovog broja poena. Neiskusni takmičari će pomisliti
da nemaju dovoljno znanja za neki zadatak koji ima veliki broj poena, pa
će se fokusirati na one sa manjim brojem poena. Ova strategija odabira 
zadataka može dovesti do zaglavljanju na zadacima određene oblasti sa kojim
takmičar ima veoma malo iskustva. Sa druge strane, takmičar može imati
znanja o nekom zadatku sa više poena, te ga u potpunosti zanemariti, jer
nije uspeo rešiti onaj sa manjim brojem poena. Izbegavanje CTF zadataka
je trenutno nerešivi problem.

%-----------------------------------------------------------------------------
\subsection{Infrastruktura zadataka}
%-----------------------------------------------------------------------------

Veb sajt se obično koristi za interakciju sa takmičarima. Veb sajt pruža 
zadatke, bodovanje, i proveru \emph{flag}-ova. Nefunkcionalni veb sajtovi
su jedan od najčešćih problema. Neki primeri: (1) nemogućnost učitavanja,
(2) ključeve je moguće pronaći grubom silom, (3) takmičari ostvaruju 
proizvoljan broj poena.

Pre samog CTF događaja organizatori treba da testiraju sve moguće
propuste infrastrukture. Serveri treba da podnesu veliki saobraćaj.
Za svaki zadatak treba imati posebnu formu koja će proveravati \emph{flag},
zajedno sa vremenskim ograničenjem provere kako bi se sprečilo korišćenje
metoda grube sile. Treba očekivati varanja, kao što se to očekuju na svim 
ostalim takmičenjima, te zbog toga treba osmisliti sisteme za detekciju 
varanja. Sve to doprinosi sigurnom i fer CTF događaju.

%-----------------------------------------------------------------------------
\subsection{Dvosmisleni zadaci}
%-----------------------------------------------------------------------------

Postoje primeri zadataka koji imaju više od jednog puta do rešenja. Isto
tako postoje zadaci kod kojih rešenja nisu jasno određena u opisu, i
sam tok rešavanja ne navodi do pravog rešenja. Ako spojimo ove dve činjenice
dobijamo dvosmislene zadatke koji se jedino mogu rešiti uz faktor sreće.
Ovakvi zadaci nastaju jer im se pridaje malo pažnje na temu i priču koju 
nosi sam zadatak. Zbog toga svaki zadatak mora imati svoj opis koji pruža
korisne informacije, zajedno sa navođenjem i uverivanjem takmičara da je na 
pravom putu.

%-----------------------------------------------------------------------------
\section{CTF na univerzitetskim kursevima}
%-----------------------------------------------------------------------------

Jedan od glavnih aspekata CTF takmičenja je u nadmudrivanju protivnika. Ali na
žalost ovaj aspekt nedostaje na mnogim univerzitetskim kursevima računarske
bezbednosti. Jedan rešenje predlažu Mirković i Piterson \cite{ctf_class} sa
svojom verzijom CTF-a: \emph{Class Capture The Flag} (CCTF). Naime, CCTF
zahteva minimalno dodatnog vremena od studenata, minimalno rada na 
osmišljanju zadata od strane profesora i saradnika u nastavi, dok u isto
vreme uključuju studente u timski rad simuliranjem realnog scenaria.
Svaki CCTF zadatak se fokusira na jednu oblast koja se obrađuje na datom 
kursu. Ovo doprinosi primeni znanja koje je student stekao tokom predavanja.
Nakon svakog CCTF događaja, vrši se analiza sa studentima o tome šta su
uradili ispravno, a šta pogrešno, sa ciljem unapređenja sledećeg CCTF 
događaja. 

%-----------------------------------------------------------------------------
\subsection{CCTF vs CTF}
%-----------------------------------------------------------------------------

CCTF se za razliku od tradicionalnog CTF-a održava na manjoj skali. Razlog 
toga je okruženje u kome se on izvršava. Naime, CCTF ima neke jedinstvene
osobine koje ga čine pogodnim za univerzitetske kurseve. Razlike izmedju
CTF-a i CCTF-a su date na tabeli \ref{tab:cctf}. U nastavku su opisane neke 
osobine koje CCTF poseduje:

\begin{table}[!h]
    \centering
    \resizebox{0.7\textwidth}{!}{\begin{minipage}{\textwidth}
        \begin{tabular}{|c|c|c|}
            \hline
            \textbf{Osobina} & \textbf{CTF} & \textbf{CCTF} \\
            \hline
            \hline
            Pripremanje & nekoliko meseci & nekoliko nedelja \\
            Trajanje & 1-2 dana & 2 časa \\
            Uloge timova & crveni ili plavi & crveni i plavi \\
            Uparivanje timova & svi na sve & parovi \\
            Učestalost & jednom godišnje & 2-3 puta po semestru \\
            Analiza & retko & uvek \\
            Težina & stručni & početni do srednji \\
            \hline
        \end{tabular}
        \end{minipage}}
    \caption{Upoređivanje CTF-a i CCTF-a.}\label{tab:cctf}
\end{table}

\begin{enumerate}
    \item \textbf{Laka realizacija.} CTF događaji zahtevaju najmanje 24 časova 
        za izvršavanje i mnogo nedelja za samu pripremu. Dok je za CCTF 
        potrebno oko dve nedelje priprema, i izvršava se u roku od 2 časa. 
        Time studenti i predavači ne gube vreme za ostale aktivnosti.
    \item \textbf{Učestaliji i sa analizom.} Mnogi CTF događaju se održavaju
        jednom godišnje i sa sobom nose malo pobednika i mnogo gubitnika.
        Ovo može da demotiviše one koji su izgubili i odvrati ih iz oblasti
        računarske bezbednosti. Mnogi učesnici nemaju visok nivo edukacije,
        i iskustva. Oni zbog toga vrlo lako odustaju od samog takmičenja.
        Sa druge strane, CCTF se dešava više puta u toku jednog semestra.
        To omogućava studentima da se takmiče više puta godišnje sa minimalno
        dodatnog ulaganja i truda. Nakon svakog CCTF vrši se analiza, gde
        učesnici mogu diskutovati o strategijama koje su koristili, i tako
        unaprede svoje veštine.
    \item \textbf{Dvostran.}
        CTF takmičenja su obično jednostrana, u smislu da timovi napadaju
        neki unapred određen sistem, ili da se brane od napada profesionalaca
        iz struke. CCTF su dvostrana, te će timovi biti u prilici da napadaju
        i da budu napadnuti. 
    \item \textbf{Doigravanje.}
        Za razliku od CTF takmičenja gde svaki tim igra protiv svih ostalih
        timova, CCTF uparuje timove u više iteracija. Ovo omogućava postojanje
        više pobedničkih timova tokom semestra i ima kao posledicu povećanje
        entuzijazma.
    \item \textbf{Svestran.}
        CCTF pruža mogućnost fokusiranja na određene oblasti, pa studenti
        mogu primeniti znanje koje su stekli tokom semestra. Sa druge strane,
        CCTF omogućava kombinaciju različitih oblasti, što će studente 
        naterati da posmatraju računarsku bezbednost u celosti. Takođe, 
        studenti će morati da donose brze odluke, i da snose posledice za one
        loše odabrane odluke.
\end{enumerate}

%-----------------------------------------------------------------------------
\subsection{Poboljšanje edukacije kroz CTF}
%-----------------------------------------------------------------------------

Gajmifikacija poboljšava pored ostalog i proces učenja. \cite{gami1, gami2}
Odatle dolazimo do prirodnog pitanja: Da li CTF (kao igra) može da poboljša
samo učenje o računarskoj bezbednosti. Postoje nekoliko rada koji se bave
ovom problematikom.

Leune i Petrilli \cite{ctf_leune} u svom radu postavljaju sledeće pitanje:
Da li se uključivanjem u realne simulacije odbrana i napada --- u obliku
CTF zadataka --- povećava efektivnost pri učenju o računarskoj bezbednosti?
Kao meru uspešnosti definišu sledeće hipoteze:

\begin{enumerate}
    \item Samopouzdanje studenata će se povećati učestvovanjem u CTF-u.
    \item Studenti će uživati u CTF-u.
    \item Studenti će steći praktične veštine učestvovanjem u CTF-u.
    \item Učestvovanje u CTF potkrepljuje teorijske koncepte.
\end{enumerate}

Prva hipoteza se pokazuje kao tačna. Mogućnost izvođenja raznih tehnika u 
kontrolisanom okruženju povećava samopouzdanje kod studenta da sam izvrši,
prepozna i odbrani se od napada. Druga hipoteza je, takođe, tačna. Mnogi
studenti su prijavili kako su veoma uživali u samom rešavanju zadataka, i
da su provodili mnogo više vremena na samim zadacima nego što su planirali.
I treća hipoteza je tačna, jer postoja jasna razlika između rezultata
onih koji su učestvovali u CTF-u i onih koji nisu. Četvrta hipoteza nije jasno
pokazana, ali se spekuliše da treba podeliti CTF zadatke na teorijski i
praktične, i da će ta podela doprineti poboljšanju samih rezultata kod
studenata.

%-----------------------------------------------------------------------------
\subsection{Prednosti i mane CTF zadataka na univerzitetskim kursevima}
%-----------------------------------------------------------------------------

CTF igre nisu idealno rešenje koje donosi samo dobre stvari sa sobom.
Kao i sve ostalo ima svoje mane. Neka zapažanja o dobrim i lošim stranama 
CTF zadataka na univerzitetskim kursevima nam daju Vikopal i dr.\ 
\cite{ctf_uni}:

\begin{enumerate}
    \item \textbf{Performanse studenata.} Postoji statistička korelacija 
        između promenljivih koje su izvučene iz CTF zadataka i 
        kolokvijum/ispita. 
    \item \textbf{Korisna navođenja.} Neka navođenja su korisnija od
        drugih. Treba izbegavati navođenja koja su očigledna i ne pružaju
        nikakvu dodatnu informaciju kako bi se problem rešio. Zbog toga ona
        moraju biti dovoljno informativna i adaptivna, ali ne smeju biti u 
        obliku uputstva.
    \item \textbf{Deljenje CTF \emph{flag}-ova.} Primećeno je četiri vrste
        plagijata: (1) slanje istih \emph{flag}-ova u bliskim vremenskim
        opsezima; (2) korektan \emph{flag} je poslat kao nekorektan
        \emph{flag} na drugom zadatku: (3) zadaci su rešene bez preuzimanja
        potrebnih fajlova; (4) brzo rešavanje uzastopnih zadataka.
    \item \textbf{CTF igre sa studentske strane.} Ankete koja je sprovedena 
        nakon uspešno završenog kursa pokazuje da većina studenata preferira
        CTF igre u odnosu na uobičajene zadatke koje imaju tokom regularnih
        kurseva.
\end{enumerate}

%-----------------------------------------------------------------------------
\section{Zaključak}
%-----------------------------------------------------------------------------

\nocite{*}

\printbibliography

\end{document}
