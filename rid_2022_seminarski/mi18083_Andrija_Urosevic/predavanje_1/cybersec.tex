%-----------------------------------------------------------------------------
% PACKAGES AND THEMES
%-----------------------------------------------------------------------------

\documentclass[aspectratio=169,xcolor=dvipsnames]{beamer}

\usetheme{SimplePlus}

\usepackage[utf8]{inputenc}
\usepackage{hyperref}
\usepackage{graphicx}
\usepackage{booktabs}

%-----------------------------------------------------------------------------
% TITLE PAGE
%-----------------------------------------------------------------------------

\title[short title]{Bezbednost Interneta}

\subtitle{Neke diskusije o zaštiti lozinki}

\author[Andrija] {Andrija Urošević}

\institute[MATF] {
Univerzitet u Beogradu\\
Matematički fakultet
}

\date{April, 2022.}

%-----------------------------------------------------------------------------
% PRESENTATION SLIDES
%-----------------------------------------------------------------------------

\begin{document}

%-----------------------------------------------------------------------------

\begin{frame}
    \titlepage
\end{frame}

%-----------------------------------------------------------------------------

\begin{frame}{Pitanja}
    \centering
    Koliko vas je hakovano?\\ \vspace{1em}
    Koliko vas je hakovalo?\\ \vspace{1em}
    Koliko vas će hakovati?
\end{frame}

%-----------------------------------------------------------------------------

\begin{frame}{Pregled}
    \tableofcontents
\end{frame}

%-----------------------------------------------------------------------------
\section{Kako čuvati lozinke?}
%-----------------------------------------------------------------------------

\begin{frame}{Čuvanje lozinki}

    \begin{alertblock}{Ključno pitanje}
        Pretpostavimo sledeći scenario: Razvijamo neku veb aplikaciju. Da bi
        naša aplikacija funkcionisala zahteva sistem prijavljivanja radi
        autentifikacije korisnika. Odlučili smo da autentifikaciju vršimo
        preko lozinke.

        \vspace{1em}

        \textit{Kako čuvati lozinke u bazi podataka?}
    \end{alertblock}

\end{frame}

%-----------------------------------------------------------------------------

\begin{frame}{Otvoreni tekst}

    \begin{block}{Otvoreni tekst}
        Otvoreni tekst u kriptografiji predstavlja nešifrovanu informaciju.
    \end{block}

    \vspace{1em}

    \begin{center}
        \begin{tabular}{c|c}
            username & password \\
            \hline
            nikola & password1 \\
            marko & matrix \\
            petar & yep59f\$4txwrr
        \end{tabular}
    \end{center}

    \vspace{1em}

    \begin{columns}
        \begin{column}{0.4\textwidth}
            \pause
            Pozitivno:
            \pause
            \begin{itemize}
                \item lako se implementira
            \end{itemize}
        \end{column}
        \begin{column}{0.4\textwidth}
            \pause
            Negativno:
            \pause
            \begin{itemize}
                \item vidljive lozinke
            \end{itemize}
        \end{column}
    \end{columns}

\end{frame}

%-----------------------------------------------------------------------------

\begin{frame}{Šifrovanje lozinki}
    \begin{block}{Šifrovanje}
        \textit{Šifrovanje} je proces enkodiranja informacija. Tokom
        šifrovanja originalna reprezentacija informacije (otvoreni tekst)
        se konvertuje u alternativnu reprezentaciju koja se naziva
        \textit{šifrat}.
    \end{block}

    \vspace{1em}

    \begin{center}
        \begin{tabular}{c|c}
            username & rot13(password) \\
            \hline
            nikola & cnffjbeq1 \\
            marko & zngevk \\
            petar & lrc59s\$4gkjee
        \end{tabular}
    \end{center}

    \vspace{1em}

    \begin{columns}
        \begin{column}{0.4\textwidth}
            \pause
            Pozitivno:
            \pause
            \begin{itemize}
                \item lozinke su skrivene
            \end{itemize}
        \end{column}
        \begin{column}{0.4\textwidth}
            \pause
            Negativno:
            \pause
            \begin{itemize}
                \item otvoreni tekst + šifrat $\implies$ ključ
            \end{itemize}
        \end{column}
    \end{columns}

\end{frame}

%-----------------------------------------------------------------------------

\begin{frame}{Heš funkcija}

    \begin{block}{Heš funkcija}
        \textit{Heš funkcija} je funkcija koja preslikava podatke proizvoljne
        veličine u vrednosti fiksirane veličine. Vrednost koju dobijemo
        primenom heš funkcije zovemo \textit{heš vrednosti} ili \textit{heš}.
    \end{block}

    \vspace{1em}

    \begin{center}
        \begin{tabular}{c|c}
            username & md5(password) \\
            \hline
            nikola & 7c6a180b36896a0a8c02787eeafb0e4c \\
            marko & 21b72c0b7adc5c7b4a50ffcb90d92dd6  \\
            petar & 47ad898a379c3dad10b4812eba843601
        \end{tabular}

    \end{center}

    \vspace{1em}

    \begin{columns}
        \begin{column}{0.4\textwidth}
            \pause
            Pozitivno:
            \pause
            \begin{itemize}
                \item heš funkcija nema inverz
            \end{itemize}
        \end{column}
        \begin{column}{0.4\textwidth}
            \pause
            Negativno:
            \pause
            \begin{itemize}
                \item \emph{rainbow} tabele
            \end{itemize}
        \end{column}
    \end{columns}

\end{frame}

%-----------------------------------------------------------------------------

\begin{frame}{Soljenje}

    \begin{block}{Soljenje}
        \textit{Soljenje} predstavlja dodavanje nasumičnih podataka ulaznom
        podatku na kome se primenjuje heš funkcija.
    \end{block}

    \vspace{1em}

    \begin{center}
        \begin{tabular}{c|c|c}
            username & salt & sha256(password+salt) \\
            \hline
            nikola & 3]iPP &
            bbe64abee254782ee5eed1e9e9...\\
            marko & PeRap & 9c09e1f111a3f526749099b9da... \\
            petar & E54Fb & f0a92d959e539c9d9711ba77da...
        \end{tabular}
    \end{center}

    \vspace{1em}

    \begin{columns}
        \begin{column}{0.4\textwidth}
            \pause
            Pozitivno:
            \pause
            \begin{itemize}
                \item \emph{rainbow} tabele ne funkcionišu
            \end{itemize}
        \end{column}
        \begin{column}{0.4\textwidth}
            \pause
            Negativno:
            \pause
            \begin{itemize}
                \item brzo računanje heš vrednosti (napad rečnikom)
            \end{itemize}
        \end{column}
    \end{columns}

\end{frame}

%-----------------------------------------------------------------------------

\begin{frame}{Dodatna poboljšanja}

    \begin{itemize}
        \item Biber tehnika (dodajemo nasumični podatak pri heširanju koji je
            zajednički za sve korisnike)
        \item Heširanje u iteracijama (nekoliko puta primenjujemo heš funkciju)
        \item Spajanje više tehnika (so + biber + heš funkcija u iteracijama)
        \item bcrypt\cite{pass3}
    \end{itemize}

\end{frame}

%-----------------------------------------------------------------------------
\section{Menadžer lozinki}
%-----------------------------------------------------------------------------

\begin{frame}{Menadžer lozinki}

    \begin{alertblock}{Ključno zapažanje}
        Kako broj usluga raste na Internetu, broj lozinki koje prosečan čovek
        treba da upamti raste respektivno, sve do trenutka kada mnogi ljudi ne
        mogu da upamte novu i jaku lozinku.

        \vspace{1em}

        Korisnici rešavaju ovaj problem na dva načina:
        \begin{itemize}
            \item Koriste istu lozinku za različite sajtove.
            \item Koriste \textit{menadžer lozinki}.
        \end{itemize}
    \end{alertblock}

    \pause

    \begin{block}{Menadžer lozinki}
        \textit{Menadžer lozinki} je program koji korisnicima pruža mogućnost
        čuvanja, generisanja, i menadžment lozinki za lokalne aplikacije ili
        Internet usluge.
    \end{block}

\end{frame}

%-----------------------------------------------------------------------------

\begin{frame}{Šta menadžer lozinki čini sigurnijim?}

    \begin{columns}

        \begin{column}{0.4\textwidth}
            \begin{enumerate}
                \item [A.1] Prednosti otvorenog k\^oda

                    \begin{itemize}
                        \item korisnici mogu prijaviti ranjive delove
                        \item smanjuje se pritisak na programere
                    \end{itemize}
                \item [A.2] Mane otvorenog k\^oda
                    \begin{itemize}
                        \item neće svi korisnici prijaviti ranjive delove
                    \end{itemize}
            \end{enumerate}
        \end{column}

        \begin{column}{0.4\textwidth}
            \begin{enumerate}
                \item [B.1] Prednosti zatvorenog k\^oda
                    \begin{itemize}
                        \item potencijalna eksploatacija ranjivosti je smanjena
                    \end{itemize}
                \item [B.2] Mane zatvorenog k\^oda
                    \begin{itemize}
                        \item korisnik mora verovati kompaniji
                        \item manji broj ljudi radi reviziju koda
                    \end{itemize}
            \end{enumerate}
        \end{column}

    \end{columns}

\end{frame}

%-----------------------------------------------------------------------------

\begin{frame}{Teorijski dizajn dobrog menadžera lozinki}
    \cite{passman1} preporučuju:

    \begin{enumerate}
            \pause
        \item Prioritet je sigurnost, a ne slučajevi upotrebe.
            \pause
        \item Otvorenog k\^oda.
            \pause
        \item Master lozinka mora biti snažna, tj. mora da zadovoljava 2017
            NIST standard.\cite{nist}
            \pause
        \item Funkcionalnost automatskog popunjavanja, zbog \emph{keylogger}-a.

            \pause
        \item Samozaključavanje nakon nekog vremena.
    \end{enumerate}

\end{frame}

%-----------------------------------------------------------------------------
\section{CTF: Capture The Flag}
%-----------------------------------------------------------------------------

\begin{frame}{CTF: Capture The Flag}

    \begin{block}{CTF (Capture The Flag)}
        \textbf{CTF} je specijalna vrsta takmičenja u oblasti računarske
        bezbednosti. Cilj takmičenja je osvojiti što više \textit{flag}-ova,
        tako što timovi pronalaze rupe u sistemu, koje onda eksploatišu.
        \textit{Flag} je obično oblika \texttt{FLAG\{neki\_tekst\}}
    \end{block}

\end{frame}

%-----------------------------------------------------------------------------

\begin{frame}{Vrste CTF-ova}

    \begin{itemize}
        \item \textbf{Jeopardy}: Timovi rešavaju unapred zadate zadatke.
        \item \textbf{Attack-Defense}: Dva tima pokušavaju da eksploatišu
            jedni druge, dok u isto vreme pokušavaju da se zaštite od napada.
        \item \textbf{Mixed}: Varijacije na temu. (na primer: wargames).
    \end{itemize}

\end{frame}

%-----------------------------------------------------------------------------

\begin{frame}{Vrste CTF zadataka}

    \begin{itemize}
        \item crypto (Kriptografija)\pause
        \item rev (Obrnuto inženjerstvo)\pause
        \item pwn (Osvojiti pristup)\pause
        \item stego (Steganografija)\pause
        \item web (Eksploatacija veb aplikacija)\pause
        \item misc (Razno)\pause
        \item hardware (Eksploatacija hardvera)\pause
        \item ...
    \end{itemize}

\end{frame}

%-----------------------------------------------------------------------------

\begin{frame}{Reference}
    \footnotesize{
        \begin{thebibliography}{99}
            \bibitem[Boonkrong, 2012]{pass1} Sirapat Boonkrong (2012)
                \newblock Security of Passwords
                \newblock \emph{Information Technology Journal} 8(2),
                112 -- 117

            \bibitem[Vaadata, 2020]{pass2} Vaadata (2020)
                \newblock How to Securely Store Passwords in Database?
                \newblock \emph{Vaadata blog article}

            \bibitem[Pravos and Mazieres, 1999]{pass3} Niels Pravos and
                David Mazieres (1999)
                \newblock A Future-Adaptable Password Scheme
                \newblock \emph{In Preceedings of 1999 USENIX Annual Technical
                Conference} 81 -- 92

            \bibitem[Kioon, Wang, Shun, Das, and Shubra, 2013]{pass4} Ah Kioon,
                Mary Cindy and Wang, Zhao Shun and Deb Das, Shubra (2013)
                \newblock Security Analysis of MD5 Algorithm in Password
                Storage
                \newblock Instruments, Measurement, Electronics and
                Information Engineering,
                \emph{Trans Tech Publications Ltd} 347(10),
                2706 -- 2711

        \end{thebibliography}
    }
\end{frame}

%-----------------------------------------------------------------------------

\begin{frame}{Reference}
    \footnotesize{
        \begin{thebibliography}{99}

            \bibitem[Luevanos, Elizarraras, Hirschi, and Yeh, 2017]{passman1}
                C. Luevanos, J. Elizarraras, K. Hirschi and J. Yeh (2017)
                \newblock Analysis on the Security and Use of Password Managers
                \newblock \emph{18th International Conference on Parallel and
                Distributed Computing, Applications and Technologies (PDCAT)},
                17 -- 24
            \bibitem[NIST 800-63A]{nist} Paul A. Grassi, James L. Fenton,
                Naomi B. Lefkovitz, Yee-Yin Choong, Jamie M. Danker,
                Mary F. Theofanos (2017)
                \newblock Digital Identity Guidelines: Enrollment and Identity
                Proofing Requirements
                \newblock \emph{NIST Special Publication 800-63A}

        \end{thebibliography}
    }
\end{frame}

%-----------------------------------------------------------------------------

\begin{frame}{Reference}
    \footnotesize{
        \begin{thebibliography}{99}

            \bibitem[CTFtime.org]{ctftime} CTFtime
                \newblock All about CTF (Capture The Flag)
                \newblock \emph{\url{ctftime.org}}

            \bibitem[OverTheWire: Wargames]{overthewire} OverTheWire
                \newblock We're hackers, and we are good-looking.
                We are the 1\%.
                \newblock \emph{\url{overthewire.org}}

        \end{thebibliography}
    }
\end{frame}

%-----------------------------------------------------------------------------

\begin{frame}

    \Huge{\centerline{\textbf{I šta ćemo sad?}}}

    \vspace{2em}

    \pause
    \small{\centerline{\url{https://capturetheflag.withgoogle.com}}}

\end{frame}

%-----------------------------------------------------------------------------

\end{document}
