\documentclass{article}

% zbog srpskog
\usepackage[utf8]{inputenc}
\usepackage[T2A]{fontenc}

% za slike 
\usepackage{graphicx}
\graphicspath{ {./images/} }

% za matematiku
\usepackage{amsmath}
\usepackage{amssymb}


\usepackage{ragged2e}

\usepackage{indentfirst}

\usepackage{biblatex} %Imports biblatex package
\addbibresource{bibliography.bib} %Import the bibliography file

\newtheorem{question}{Питање}
\newtheorem{image}{Слика}


\title{Етика коришћења вештачке интелигенције за препознавање емоција у дечјим играчкама}
\author{Јован Милић, 127/2017 }
\date{Мај 2022.}

\begin{document}

\maketitle

\justifying
\section{Увод}


Емоционална вештака интелигенција се односи на технологије које \\користе технике вештачке интелигенције да би се осетиле и препознале људске емоције, као и да би се интераговало са њима. 

Овај рад се управо бави порастом употребе емоционалне вештачке \\интелигенције у дечјим играчкама и другим услугама које су доступне деци. Тачније, испитује се етика употребе оваквих технологија, на који начин би њима требало управљати и регулисати их, услови (уколико постоје) у којима је прихватљиво користити их, као и какво је мишљење родитеља о свему томе.  

\section{Историја и развој играчака}
Иако можда изгледа као да је емоционална вештачка интелигенција новина у свету играчака, постоји историјски контекст. Наиме, играчке које имају роботске карактеристике се појављују јако рано. 

На пример, 1920-их година се појављује фонографска лутка под називом "Dolly Rekord" и имала је могућност да рецитује дечје песме. Године 1959. компанија Мател је направила лутку "Chatty Cathy" која је могла да \\изговори 11 фраза укључујући "Волим те". Седамдесетих година лутке које се производе садрже транзисторе и могу да репродукују претходно снимљене електронске поруке. Развој микропроцесора омогућава и развој интерактивних играчака као што је "Teddy Ruxpin" која може да чита дечје приче. 

\includegraphics[scale=0.4]{dolly_rekord}
\begin{image}
\centering
"Dolly Rekord" лутка
\end{image}

Деведесете године доносе нови напредак. Насупрот дотадашњим \\играчкама на које су деца пројектовала своје емоције, "паметне играчке" уносе нов облик интеракције између деце и играчака и захтевају да деца буду свесна емотивног стања играчке. Овакве играчке су, иако су на \\почетку биле једноставног дизајна, изазвалe нека нова питања, као што је питање пожељности да се синтетички ентитети усреће и да им се намире потребе. 

Данас је њихов дизајн много напреднији и порасла је комплексност таквих уређаја јер се сада ослањају и на повезаност на интернет и умре-жавање, као и на употребу различитих сензора и препознавање слика и гласова. 

Иако су овакве врсте играчака још увек релативно ретке у домовима, оне постају све доступније и распрострањеније. Добар пример модерне играчке данашњице јесте кућни социјални робот Козмо. Козмо има могућности да учи, прилагођава се и има своја "расположење". Његово расположење опадне ако је гладан, а добија самопоуздање ако победи у игри. Уз помоћ камера Козмо може да препозна лица и научи имена људи, а такође користи и емоционалну вештачку интелигенцију да реагује на основне емоције: бес, гађење, страх, срећу, тугу и изненађење.

У блиској будућности се очекује да овакве играчке имају још више могућности, али треба нагласити да њихов напредак не зависи само од напретка технологије већ и од тога до које мере треба допустити испре-плетаност са објектима, нарочито у раним фазама живота. 


\includegraphics[scale=0.6]{cozmo}
\begin{image}
\centering
Социјални робот Козмо
\end{image}


\section{Етичка питања}
Иако напредак технологије доноси велики број погодности и унапређује квалитет живота, постоје и проблеми који због тога настају и чијем се решавању и уклањању треба посветити. Конкретно на примеру употребе вештачке интелигенције у производњи играчака за децу, теме које се намећу су право на приватност,  као и потреба за добрим управљањем и регу-лисањем.

Велики недостатак је тај што деца имају мало контроле над претварањем њиховог детињства у податке. Свест деце о таквим стварима, као што је и очекивано, је премала. 
"Када твоја сопствена игра и твојој спаваћој соби није више приватна, када твој дом није више приватан, када твоја осећања и емоције нису више твоје да их делиш када то одабереш, већ се анализирају и деле са другима чак и када ниси спреман да их делиш - мислим да смо тој деци учинили огромну медвеђу услугу". \cite{mcstay2021emotional}

Емоционална вештачка интелигенција појачава информациону \\асиметрију услед неспособности деце да уклопе своје емоције у технологију. Насупрот томе, одрасли имају могућност да "преговарају" са технологијама  емоци-оналне вештачке интелигенције: могу да одлуче да не корсите \\одређену технологиjу, или да искључе функционалност регистровања \\емоција на тим уређајима.

Природно се поставља питање шта заправо подразумева дечја писменост о технологији и подацима. Ако кажемо да је се то односи на знање, вештине и ставове који дозвољавају деци да сазревају и да се развијају у дигиталном свету који константно расте, тада изазов да разумемо како се емоције кате-горизују постаје непремостив. Још један битан аспект који треба размо-трити јесте трајност података, као и право на прошлост. Деца морају да имају право да забораве на делове свог детињства, а употреба оваквих технологија то знатно отежава. Повећање способности конверзације и \\комуникације код играчака доводи до тога да деца у њима могу да виде пријатеље и да им се поверавају. Ово се граничи са фундаменталним начелима приватности, јер делови одрастања као што су: физички, ме-нтални, духовни, морални и социјални развој заслужују да буду само у сећањима детета и његове породице јер само они могу да придају прави значај томе. Није им место у некој бази података где се користе у сврхе маркетинга.

\subsection{Регулисање законом}
Управо из наведених разлога је јасно да мора да постоје недвосмислении презицно дефинисани закони и регулација употребе података које паметне играчке и слични уређаји сакупљају. Оно што је овде проблематично је то што су емоције нешто апстрактно и самим тим их је тешко дефинисати и уклопити у законске формате. На пример, Општа уредба о заштити података Европске уније (ГДПР) нема никакву референцу на емоције. \\Слично, предлог за уредбу о Е-приватности ретко спомиње емоције. Једино се у два дела преамбуле помињу емоције, али се ту дефинишу као изузетно осетљиве.  Још један проблем који се јавља код ГДПР-а је проблем тери-торије. Наиме, примена овог закона се одвија и изван граница ЕУ, јер се односи на субјекте који се налазе у Европксој унији (а не само на њене грађане). Правни стручњаци у Европи наглашавају да су се прве паметне играчке често производиле (и даље се производе) у Сједињеним Америчким \\Државама или у Кини, где су на снази другачији закони који регулишу заштиту података. Установити да ли су подаци о људским емоцијама лични подаци или осетљиви лични подаци није нимало лак задатак, као ни утвр-дити где је та граница и у ком тренутку се ти подаци уклапају у једну или у другу категорију. Са друге стране, такве кораке је неопходно предузети да би уредбе о заштити података биле потпуније. 

\subsection{Давање пристанка}
Када је у питању регулисање приватности података, кључни аспект јесте пристанак. Међутим, то је осетљиво када се ради о деци, јер се поставља питање са колико година би они били у стању да дају свој пристанак. ГДПР поставља границу за године када је у питању давање пристанка на 16, али дозвољава државама чланицама да ову границу спусте до 13 година. \\
Важно је споменути и да је битан и тренутак у ком се тражи пристанак: да ли се то одвија у радњи пре куповине, да ли је написано на кутији, или се налази у њој?

\section{Мишљење родитеља}
Иако су централна тема овог рада деца, очигледно је да се морамо дотаћи ставова и мишљења њихових родитеља. У наставку ће бити при-казани и анализирани резултати истраживања које су спроводили Ендру МекСтеј и Гилад Рознер у вези са прихватљивошћу употребе вешчтаке емоционалне интелигенције у технологијама које се фокусирају на децу. 

\subsection{Прво питање}
\begin{question}
Колико вам је пријатна идеја о играчкама за децу повезаних на интернет која обрађују податке о дејчим емоцијама?
\end{question}

Родитељима нису дата детаљна објашњења какве су технологије у пи-тању или како функционишу, већ је од њих тражено да размотре принцип паметних играчака.

\includegraphics[scale=0.3]{graph1}
\begin{image}
\centering
Резултати одговора на прво питање
\end{image}

Резултати показују да то код родитеља не изазива ни велику лагодност ни нелагодност. Да им ова идеја изазива непријатност у некој количини се изјаснило 48\% насупрот 30\% који замисао мање-више сматрају пријатном. Ово указује да иако не изазива пријатна осећања код неких родитеља, има и оних који прихватају идеју, па се може рећи да је резултат у неку руку уравнотежен.

\subsection{Друго питање}
\begin{question}
Колико би вам било пријатно да свом детету дате ручни сат који је повезан на интернет, или неки уређај који могу да носе који би вас извештавао о емоционалном стању детета: да ли су срећни, под стресом, бесни или тужни? Овакав уређај би давао информације на дневном нивоу, као и дугорочнији преглед њиховог емоционалног стања током одређеног временског периода.
\end{question}
\includegraphics[scale=0.3]{graph2}
\begin{image}
\centering
Резултати одговора на друго питање
\end{image}

Највећи проценат испитаника је одговорио са "Веома пријатно", мада треба напоменути да одмах за њим иде одговор "Веома непријатно". 

Генерално је 37\% одговорило да им је у некој мери непријатно, док је супротног мишљења било 43\%. Слично као код првог питања може се приметити да су мишљења подељења и да не постоји јако слагање око једног одговора.


\subsection{Треће питање}
\begin{question}
У којој мери се слажете или не слажете са следећом изјавом: "Компаније које производе играчке које бележе и користе податке о дечјим осећањима (од фацијалних експресија, гласова, рада срца и других података о телу) имају дужност да се обрате одговарајућим  властима уколико мисле да је дете можда злостављано, да се само-повређује или је на из неких других разлога под превеликим стресом".
\end{question}

Ово питање је мотивисано потребом из подручја технолошке етике да се разграничи разум човека од способности машина, а такође има и додирних тачака са питањем регулисања употребе оваквих технологија законом.

\includegraphics[scale=0.3]{graph3}
\begin{image}
\centering
Резултати одговора на треће питање
\end{image}

Овде је ситуација мало другачија у односу на претходна два питања. Наиме, овде се испитаници већински слажу у некој мери са наведеном изјавом. Тачније, 54\% се слаже, док се само 16\% не слаже са изреченим. То и јесте очекивана реакција јер би таква употреба података могла да донесе много користи. Међутим, треба обазриво размотрити такву идеју, првенствено зато што је тема злостављања деце јако осетљива, а простор за грешку при анализи оваквих податка је велики.

\newpage
\section{Закључак}
Емоционална вештака интелигенциjа подразумева коришћење вештачке 
интелигенциje како би се осетиле и препознале људске емоциjе, као и да би се интераговало са њима. Играчке које имају способност да интерагују са децом на неки начин нису новина, али са напретком технологије расле су и њихове могућности. Такве паметне играчке се данас ослањају на интернет и употребу различитих сензора. 

Очигледно је да велики напредак повлачи са собом и потребу за већом контролом и регулисањем таквих уређаја. Пред нама као друштвом није нимало лак задатак, јер то покреће многа етичка питања, нарочито зато што се ради о деци. Прво и основно питање које се намеће је питање приватности података, а баш зато што се ради о овако осетљивој групи корисника то питање није уопште лако решити, као што би то био случај да су у питању одрасли и зрели људи. Још један важан аспект је и трајност тих података и право деце да забораве делове свог детињства. 

Сви ови проблеми су нешто што дигитално доба носи са собом, те је због тога велика одговорност на родитељима да не прибегавају оваквим технологијма због недостатка искуства у васпитању детета, или зато што немају времена да им се посвете. Са друге стране, мишљења сам да закони и уредбе који регулишу употребу података морају да буду прецизнији како би се извукао максимум из онога што нам технолошки напредак нуди.

\newpage
\printbibliography[title={Литература}]

\end{document}
